%
%  untitled
%
%  Created by Andrew Tulloch on 2009-11-04.
%  Copyright (c) 2009 __MyCompanyName__. All rights reserved.
%!TEX TS-program = xelatex
%!TEX encoding = UTF-8 Unicode

\documentclass[10pt, oneside, reqno]{amsart}
\usepackage{geometry, setspace, graphicx, enumerate}
\onehalfspacing                 
\usepackage{fontspec,xltxtra,xunicode}
% \defaultfontfeatures{Mapping=tex-text}
% 
% \setromanfont[Mapping=tex-text,Contextuals= 
% {NoWordInitial,NoWordFinal,NoLineInitial,NoLineFinal}]{Hoefler Text}
% \setsansfont[Scale=MatchLowercase,Mapping=tex-text]{Hoefler Text}
% \setmonofont[Scale=MatchLowercase]{Andale Mono}


    % AMS Theorems
    \theoremstyle{plain}% default 
    \newtheorem{thm}{Theorem}[section] 
    \newtheorem{lem}[thm]{Lemma} 
    \newtheorem{prop}[thm]{Proposition} 
    \newtheorem{cor}[thm]{Corollary} 

    \theoremstyle{definition} 
        \newtheorem{defn}[thm]{Definition}
        \newtheorem{conj}[thm]{Conjecture}
        \newtheorem{exmp}[thm]{Example}
    
    \theoremstyle{remark} 
        \newtheorem*{rem}{Remark} 
        \newtheorem*{note}{Note} 
        \newtheorem{case}{Case} 

\newcommand{\al}{\alpha}
\newcommand{\Q}{\mathbb{Q}}
\newcommand{\R}{\mathbb{R}}
\newcommand{\Com}{\mathbb{C}}
\newcommand{\Z}{\mathbb{Z}}
\newcommand{\F}{\mathbb{F}}

\newcommand{\Ga}{\mathbb{G}}

\newcommand{\iprod}[2]{\langle #1, #2 \rangle}

\newcommand{\met}{(X,d)}
\newcommand{\topo}{(X,\tau)}


\newcommand{\aut}[1]{\text{Aut}{(#1)}}

\newcommand{\gener}[1]{\langle #1 \rangle}
\newcommand{\charr}[1]{\text{char}(#1)}
\newcommand{\nth}{n\textsuperscript{th}}

\newcommand{\tworow}[2]{\genfrac{}{}{0pt}{}{#1}{#2}}
\newcommand{\xdeg}[2]{[#1 : #2]}
\newcommand{\gal}[2]{\text{Gal}(#1/#2)}
\newcommand{\minpoly}[2]{m_{#1, #2}(x)}

\newcommand{\mapping}[5]{\begin{align*}
    #1 : \quad     #2 &\rightarrow #3 \\
            #4  &\mapsto #5
\end{align*}}

\newcommand{\intr}{\text{Int}\,} 
\newcommand{\ol}[1]{\overline{#1}} 
\newcommand{\xbf}{X \,\backslash \, F}

        
\usepackage{hyperref}
        
\title{MATH 3961 - Metric Spaces}                               % Document Title
\author{Andrew Tulloch}
%\date{}                                           % Activate to display a given date or no date


\begin{document}
\maketitle \tableofcontents \clearpage



\section{Metric Spaces} % (fold)
\label{sec:metric_spaces}

\begin{defn}[Metric]
    A metric, or distance function, on a set $X$ is a mapping $d : X \times X \rightarrow \R$ such that 
    \begin{itemize}
        \item $d(x,y) \geq 0$ for all $x,y \in X$, and $d(x,y) = 0$ if and only if $x = y$.
        \item $d(x,y) = d(y,x)$ for all $x,y \in X$.
        \item $d(x,z) \leq d(x,y) + d(y,z)$ for all $x,y,z \in X$.
    \end{itemize}
    
    We call $(X,d)$ a \textbf{metric space}.
\end{defn}

\begin{defn}[Open ball]
    Let $(X,d)$ be a metric space.  For $x \in X$ and $\epsilon > 0$, the set $B_d(x, \epsilon)$ defined by \[
        B_d(x,\epsilon) = \{ y \in X \, | \, d(x,y) < \epsilon \}
    \]
    is callen an \textbf{open ball} in the set $X$.  
\end{defn}

\begin{defn}[Open sets in metric spaces]
    Let $(X,d)$ be a metric space and let $U$ be any subset of $X$.  Then $U$ is called an \textbf{open set} in $X$ if every point of $U$ is an interior point of $U$; that is, for any $a \in U$, there is an open ball $B(a, \epsilon)$ such that $B(a, \epsilon) \subseteq U$.
\end{defn}

\begin{defn}[Properties of open sets]
    Let $\met$ be a metric space.
    \begin{itemize}
        \item $\emptyset$ and $X$ are open.
        \item The union of an arbitrary collection of open sets is open.
        \item The intersection of a \textbf{finite} number of open sets is open.
    \end{itemize}
\end{defn}




\begin{defn}[Closed set]
    A subset $A$ of a metric space $\met$ is \textbf{closed} if it's complement $X \backslash A$ is open in $X$.
\end{defn}

\begin{defn}[Properties of closed sets]
    Let $\met$ be a metric space.
    \begin{itemize}
        \item $\emptyset$ and $X$ are closed.
        \item The union of an finite collection of closed sets is closed.
        \item The intersection of an arbitrary number of closed sets is closed.
    \end{itemize}
\end{defn}

\begin{defn}[Limit point of a subset]
    Let $\met$ be a metric space and let $A$ be a subset of $X$.  Then a point $x$ in $X$ is a \textbf{limit point} of $A$ if every open ball $B(x, \epsilon)$ contains at least one point of $A$.
    
    The set of all limit points of $A$ is called the \textbf{derived set} $A'$.
\end{defn}

\begin{defn}[Closure of a set]
    Let $\met$ be a metric space and let $A \subseteq X$.  Then the set consisting of $A$ and its limit points is called the \textbf{closure} of $A$, denoted $\ol{A}$.  \[
        \ol{A} = A \cup A'
    \]
\end{defn}

\begin{thm}
    The closure of a set is a closed set, and a set is closed if and only if it is equal to its closure.
\end{thm}

\begin{defn}[Interior of a set]
    Let $\met$ be a metric space and let $A \subseteq X$.  A point $a \in A$ is an \textbf{interior point} of $A$ if there exists $\epsilon > 0$ such that \[
        B(a, \epsilon) \subseteq A
    \]
    
    The set of interior points of $A$ is called $\intr A$, the \textbf{interior} of $A$.
\end{defn}

\begin{thm}
    The set $\intr A$ is open, and a set $A$ is open if and only if $\intr A = A$.
\end{thm}

\begin{thm}[Properties of interior and closure]
    The interior of a set $A$ is the largest open subset contained in $A$, and the closure of $A$ is the smallest closed set containing $A$.
\end{thm}


\begin{defn}[Isolated point]
    Let $\met$ be a metric space and let $A$ be a subset of $X$.  A point $x \in A$ is called an \textbf{isolated point} if there exists an $\epsilon > 0$ such that \[
        B(x, \epsilon) \backslash \{ x \} \cap A = \emptyset
    \]
\end{defn}

\begin{defn}[Boundary of a subset]
Let $\met$ be a metric space and $A \subseteq X$.  Then the \textbf{boundary} of $A$ is defined as \[
    \partial A = \ol{A} \cap \ol{X \backslash A} = \ol{A} \backslash \intr A
\]
\end{defn}
\begin{thm}[Properties of the boundary]
    Let $\met$ be a metric space and $A \subset X$. Then we have \begin{itemize}
        \item $\ol A = \intr A \cup \partial A$.
        \item $A$ is closed if and only if $\partial A \subseteq A$.
        \item $A$ is open if and only if $\partial A \subseteq X \backslash A$.
        \item $\partial (X \backslash A) = \partial A$.
    \end{itemize}
\end{thm}


\begin{defn}[Diameter of a set]
    The diameter of a subset $A$ of a metric space $\met$, $\delta(A)$ is defined as \[
        \delta(A) = \sup_{x,y \in A} d(x,y)
    \]
\end{defn}

\begin{defn}[Bounded set]
    A subset $A$ of a metric space $\met$ is \textbf{bounded} if its diameter is finite.  Alternatively, a subset is bounded if it is contained in a large enough open set - i.e., there exists $x \in X$ and $\epsilon > 0$ such that $ A \subseteq B(x, \epsilon)$ 
\end{defn}

\subsection{Separable metric spaces} % (fold)
\label{sub:separable_metric_spaces}


\begin{defn}[Separable metric space]
    Let $\met$ be a metric space.  Then a subset $A$ of $X$ is said to be \textbf{dense} in $X$ if $\ol A = X$.  A metric space $\met$ is said to be \textbf{separable} if $X$ has a countable dense subset.
\end{defn}

\begin{cor}
    We note that $A$ is dense in $X$ if and only if for any $x \in X$ and $\epsilon > 0$, there is a point $a \in A$ such that $d(x,a) < \epsilon$.
\end{cor}

% subsection separable_metric_spaces (end)

\subsection{Subspaces} % (fold)
\label{sec:subspaces}

\begin{defn}[Open sets in a subspace]
    Let $\met$ be a metric space and $(Y, d_Y)$ be a metric subspace of $\met$.  Let $G$ be a subset of $Y$.  Then $G$ is open in $Y$ if and only if, for any $x \in G$, there is an open ball $B(x, \epsilon)$ in $X$ such that \[
        B(x, \epsilon) \cap Y \subseteq G
    \]
    A subset $H$ of $Y$ is closed in $Y$ if its complement $G = Y \backslash H$ of $H$ is open in $Y$.

\end{defn}
\begin{thm}[Open sets in a metric subspace]
    Let $(Y, d_Y)$ be a metric subspace of a metric space $\met$, and let $G \subseteq Y$.  Then $G$ is open in $Y$ if and only if there exists an open subset $U$ in $X$ such that $G = U \cap Y$.
\end{thm}



% subsection subspaces (end)

\subsection{Convergence in a Metric Space} % (fold)
\label{sub:convergence_in_a_metric_space}

\begin{defn}[Convergence]
    A sequence $(x_n)$ in a metric space $\met$ is said to \textbf{converge} to a point $x \in X$ if for any $\epsilon > 0$, there exists $N$ such that \[
        n > N \quad \text{implies} \quad d(x_n, x) < \epsilon
    \]
    The point $x$ is called a \textbf{limit} of the sequence $(x_n)$
\end{defn}

\begin{cor}
    A sequence $(x_n)$ in a metric space $\met$ is said to \textbf{converge} to a point $x \in X$ if any open ball $B(x, \epsilon)$ contains almost all $x_n$.
\end{cor}

\begin{thm}[Connection between closed sets and convergent sequences]
    Let $\met$ be a metric space, $A \subseteq X$ and $x \in X$.  Then 
    \begin{itemize}
        \item $x \in \ol{A}$ if and only if there is a sequence $(x_n)$ in $A$ such that $x_n \rightarrow x$.
        \item $A$ is closed if and only if $A$ contains all the limits of convergent sequences in $A$.
    \end{itemize}
\end{thm}

\begin{defn}[Uniform convergence]
Let $(f_n)$ be a sequence of real-valued functions defined on a set $S$ and let $f$ be a function defined on $S$.  Then we say that the sequence $(f_n)$ converges to $f$ uniformly if for any $\epsilon > 0$, there exists $N$ such that \[
    \sup_{x \in S} d(f_n(x), f(x)) < \epsilon
\] for all $n > N$, and where $N$ is independent of $x$.
\end{defn}

\begin{defn}[Cauchy Sequences]
    A sequence $(x_n)$ in a metric space $\met$ is said to be \textbf{Cauchy} in $X$ if for any $\epsilon > 0$, there exists $N$ such that \[
        m,n > N \Rightarrow d(x_m, x_n) < \epsilon
    \]
\end{defn}

\begin{defn}[Completeness in Metric Spaces]
A space $X$ is said to be complete if every Cauchy sequence in $X$ converges in $X$.
\end{defn}

\begin{prop}
    Every convergent sequence $(x_n)$ in a metric space $\met$ is a Cauchy sequence.
\end{prop}

\begin{cor}
    Let $\met$ be a complete metric space.  Then a closed metric subspace $Y = (Y, d_Y)$ of $X$ is complete.  
\end{cor}

\begin{prop}
    Let $\met$ be a metric space.  If a Cauchy sequence $(x_n)$ in $\met$ has a subsequence converging to $x$, then $(x_n)$ converges to $x$.
\end{prop}
% subsection convergence_in_a_metric_space (end)

% section metric_spaces (end)



\section{Continuous Mappings} % (fold)
\label{sec:continuous_mappings}

\begin{defn}[Continuous mapping between metric spaces]
    Let $\met$ and $(Y, d_Y)$ be two metric spaces.  Then a mapping $f : X \rightarrow Y$ is said to be \textbf{continuous} at a point $a \in X$ if for any $\epsilon > 0$, there exists $\delta$ such that \[
        d_X(x,a) < \delta \Rightarrow d_Y(f(x), f(a)) < \epsilon
    \]
\end{defn}

\begin{thm}[Topological characterisations of continuity]
    A funtion $f : X \rightarrow Y$ is continuous at $a \in X$ if and only if for any open set $W$ containing $f(a)$, there exists an open set $G$ containing $A$ such that $f(G) \subseteq W$.  
\end{thm}

\begin{thm}[Sequential characterisation of continuity]
    A function $f: X \rightarrow Y$ is continuous at $a \in X$ if and only if for any sequence $(x_n)$ which converges to $a$ in $X$, the corresponding sequence $(f(x_n))$ converges to $f(a)$ in $Y$.
\end{thm}


\begin{defn}[Continuous mapping]
    A map is continuous on $X$ if any only if it is continuous at every point in $X$.
\end{defn}

\begin{thm}[Topological definition of continuous mapping]
    A mapping $f: X \rightarrow Y$ is continuous on $X$ if and only if for any open set $W$ in $Y$, the set $f^{-1}(W)$ is open in $X$. Alternatively, a function is continuous if the preimage of open sets are open in $X$.
\end{thm}

\begin{thm}[Continuity in terms of closed sets]
    A mapping $f: X \rightarrow Y$ is continuous on $X$ if and only if the preimage of closed sets are closed in $X$.
\end{thm}

% section continuous_mappings (end)



\section{Homeomorphism and Equivalent Metrics} % (fold)
\label{sec:homeomorphism_and_equivalent_metrics}
\begin{defn}[Homeomorphism]
    Let $X$ and $Y$ be metric spaces and let $f: X \rightarrow Y$ be a map between them.  Then $f$ is a \textbf{homeomorphism} from $X$ to $Y$ if we have \begin{itemize}
        \item $f$ is a bijection.
        \item $f$ and $f^{-1}$ are continuous
    \end{itemize}

    If a homeomorphism exists between $X$ and $Y$, we say that $X$ and $Y$ are \textbf{homeomorphic}, and that $X \simeq Y$.
\end{defn}

\begin{defn}[Characterisations of homeomorphism]
Let $f: X \rightarrow Y$ be a bijective mapping.  Then the following are equivalent.
\begin{itemize}
    \item $f$ is a homeomorphism;
    \item for any $U \subseteq X$, $U$ is open in $X$ if and only if $f(U)$ is open in $Y$;
    \item for any $G \subseteq X$, $G$ is closed in $X$ if and only if $f(G)$ is closed in $Y$;
    \item for any $A \subseteq X$, $f(\ol{A}) = \ol{f(A)}$;
    \item for any $B \subseteq Y$, $\ol{f^{-1}(B)} = f^{-1}(\ol{B})$
    \item for any $B \subseteq Y$, $f^{-1}(\intr B) = \intr f^{-1}(B)$
\end{itemize}
\end{defn}

\begin{defn}[Isometric mappings]
    Let $\met$ and $(Y, d_Y)$ be two metric spaces and $f: X \rightarrow Y$ a mapping.  Then $f$ is said to be \textbf{isometric} or an \textbf{isometry} if $f$ preserves distances; that is, for all $x,y \in X$, \[
        d_Y(f(x), f(y)) = d_X(x,y)
    \]
    The space $X$ is said to be isometric with the space $Y$ if there exists a bijective isometry of $X$ onto $Y$.  The spaces $X$ and $Y$ are isometric spaces
\end{defn}

\begin{thm}
    Any isometric mapping from $X$ onto $Y$ is a homeomorphism.  Moreover, if $X$ is complete and $Y$ is isometric with $X$, then $Y$ is also complete.
    \end{thm}
    
\begin{defn}[Equivalent metrics]
    Let $(X, d_1)$ and $(X, d_2)$ be two metric spaces.  If the identity mapping $\text{id} : (X, d_1) \rightarrow (X, d_2)$ is a homeomorphism, then the metrics $d_1$ and $d_2$ are said to be equivalent on $X$.
\end{defn}

\begin{thm}[Characterisations of equivalent metrics]
    Let $(X, d_1)$ and $(X, d_2)$ be two metric spaces.  Then the following are equivalent.
    \begin{itemize}
        \item The metrics $d_1$ and $d_2$ are equivalent on $X$;
        \item for any $U \subseteq X$, $U$ is open in $(X, d_1)$ if and only if $U$ is open in $(X, d_2)$;
        \item for any $G \subseteq X$, $G$ is closed in $(X, d_1)$ if and only if $G$ is closed in $(X, d_2)$;
        \item The sequence $(x_n)$ converges to $a$ in $(X, d_1)$ if and only if it converges to $a$ in $(X, d_2)$.
    \end{itemize}
\end{thm}

\begin{thm}[Equivalent metrics]
    Let $(X, d_1)$ and $(X, d_2)$ be two metric spaces.  If there exist strictly positive numbers $c$ and $C$ such that \[
        c d_1(x,y) \leq d_2(x,y) \leq C d_1(x,y)
    \] for all $x,y \in X$, then the metrics $d_1, d_2$ are equivalent on $X$.
    \end{thm}
% section homeomorphism_and_equivalent_metrics (end)

\section{Contraction Mapping Theorem} % (fold)
\label{sec:contraction_mapping_theorem}

\begin{defn}[Uniformly continuous]
    A function $f: X \rightarrow Y$ is uniformly continuous if for any $\epsilon > 0$ there exists $\delta > 0$ such that \[
        d_X(x,y) < \delta \Rightarrow d_Y(f(x), f(y)) < \epsilon
    \] where $\delta$ is independent of $x,y$.
\end{defn}

\begin{defn}[Contraction mapping]
    Let $f$ be a mapping from a metric space $\met$ to itself.  Then $f$ is called a \textbf{contraction mapping} if there exists a constant $K > 1$ such that for all $x,y \in X$, \[
        d(f(x), f(y)) \leq K (x,y)
    \] 
\end{defn}

\begin{prop}
    If $f: X \rightarrow X$ is a contraction mapping, then $f$ is \textbf{uniformly continuous}, and hence continuous, on $X$
\end{prop}



\begin{thm}[Banach fixed point theorem]
    Let $\met$ be a complete metric space and let $f: X \rightarrow X$ be a contraction mapping.  Then $f$ has a unique fixed point $p$ in $X$
\end{thm}

\begin{proof}
    \textbf{Existence:} Show that the sequence $(x_n)$, defined as $x_n = f^n(x_0)$ for some $x_0 \in X$ is Cauchy.
    
    \textbf{Uniqueness:} If $p$ and $q$ are fixed points of $f$, the we have that \[
            d(p,q) = d(f(p), f(q)) \leq K d(p,q)
            \] and so $p = q$.
\end{proof}

\begin{cor}[Application of Banach fixed point theorem to ordinary differential equations]
    We seek to solve the differential equation \[
        \frac{dx}{dt} = f(t,x)
    \] given the initial condition $x(t_0) = x_0$.
    
    Define $Y$ be the subspace of the set of all continuous functions on $[t_0 - \beta, t_0 + \beta]$ with the supremum metric, satisfying $d(x(t), x_0) < c \beta$
    We claim that the mapping $F: Y \rightarrow Y$ defined by\[
        F(x(t)) = x_0 + \int_{t_0}^t f(s, x(s)) \, ds
    \] is a contraction mapping.  As $F$ is a contraction mapping on a complete metric space, we must have that it has a unique fixed point, which satisfies the differential equation above.
\end{cor}

% section contraction_mapping_theorem (end)

\section{Completeness} % (fold)
\label{sec:completeness}

We recall the definition of completeness in metric spaces.

\begin{defn}[Completeness in Metric Spaces]
A space $X$ is said to be complete if every Cauchy sequence in $X$ converges in $X$.
\end{defn}

We now state the theorem that every metric space can be completed. The space $\hat{X}$ in the theorem is called the completion of the given space $X$.

\begin{thm}[Completion of a metric spaces]
    Let $\met$ be a metric space.  Then there exists a complete metric space $\hat{X} = (\hat{X}, \hat{d})$ which has a subspace $W$ that is isometric with $X$ and is dense in $\hat{X}$
\end{thm}

\begin{proof}
    TO DO
\end{proof}
% section completeness (end)

\section{Connectedness} % (fold)
\label{sec:connectedness}


\begin{defn}[Disconnected]
Let $X$ be a metric space (or topological space). Then $X$ is said to be \textbf{disconnected} if there exist two non-empty subsets $A_1, A_2$ of $X$ such that \[
    X= A_1 \cup A_2 \text{ and } A_1 \cap \ol A_2 , \ol A_1 \cap A_2 = \emptyset
\]

If no two sets $A_1, A_2$ exist, we say that $X$ is \textbf{connected}
\end{defn}

\begin{thm}[Characterisations of connectedness]
    Let $\met$ be a metric space. Then the following statements are equivalent:
    \begin{itemize}
        \item $X$ is disconnected;
        \item There exist two non-empty disjoint open subsets $A_1, A_2$ in $X$ such that $X = A_1 \cup A_2$;
        \item There exist two non-empty disjoint closed subsets $A_1, A_2$ in $X$ such that $X = A_1 \cup A_2$; 
        \item There exists are proper subset of $X$ which is both open and closed in $X$
    \end{itemize}
\end{thm}

\begin{defn}[Connected subspace]
    Let $\met$ be a metric space and $A$ a non-empty subset of $X$.  Then $A$ is said to be a connected subst of $X$ if $A$ is connected as a metric subspace and to be a \textbf{disconnected} subset of $X$ if $A$ is disconnected as a metric subspace.
\end{defn}

\begin{thm}[Intervals in $\R$]
    A subset $A$ of $\R$ containing at least two points is connected if and only if $A$ is a interval.
\end{thm}

\begin{thm}[Characterisations of connectedness]
    Let $\mathcal{S}(2)$ be the two point discrete metric space.
    If $A$ is connected then any continuous mapping $f: A \rightarrow \mathcal{S}(2)$ is a constant mapping.  Alternatively, if $f : A \rightarrow \mathcal{S}(2)$ is continuous, then $f(A) = \{0\}$ or $\{1\}$.
\end{thm}

\begin{thm}[Connectedness is a topological property]
    Let $X$ and $Y$ be two metric spaces, and let $f: X \rightarrow Y$ be a continuous mapping.  Then if $A \subseteq X$ is connected in $X$, then the image $f(A)$ is connected in $Y$.
\end{thm}


\begin{defn}[Path-connected]
    Let $X$ be a metric space and $A$ a subset of $X$. Then $A$ is said to be \textbf{path-connected} if for any $a,b \in A$, there is a path joining $a$ and $b$, that is, a continuous mapping $f: [0,1] \rightarrow A$ such that $f(0) = a, f(1) = b$.
\end{defn}



\begin{thm}[Path-connectedness implies connectedness]
    Let $X$ be a metric space and $A$ a subset of $X$.  If $A$ is path-connected, then $A$ is connected. 
\end{thm}
We note that the converse is not necessarily true - that is, there exist connected sets that are not path connected.  However, in $\R^n$, we have the following.

\begin{thm}[For open sets in $\R^n$, path-connectedness is equivalent to connectedness]
    Let $X$ be any open set in $\R^n$.  Then $X$ is connected if and only if $X$ is path-connected.
\end{thm}
% section connectedness (end)




\section{Compactness} % (fold)
\label{sec:compactness}
Compactness in metric spaces.

\begin{defn}[Open covering of a set]
    Let $X$ be a metric space (or any topological space), and let $A \subseteq X$.  Then a family $\mathcal{U}$ of open sets in $X$, is called an \textbf{open covering} of $A$ if \[
        A \subseteq \cup_{U \in \mathcal{U}} U
    \]
A subset $\mathcal{V}$ of $\mathcal{U}$ is called a \textbf{finite subcovering} if $\mathcal{V}$ covers $A$ and has a finite number of elements.
\end{defn}

\begin{defn}[Compact subset of a topological space]
    let $X$ be a metric space (or any topological space), and let $A \subseteq X$.  Then $A$ is called a \textbf{compact subset} of $X$ if \textbf{every} open covering $\mathcal{U}$ of $A$ has a finite subcovering $\mathcal{V}$ of $A$. 
\end{defn}

In metric spaces, we have the following useful theorem.

\begin{thm}[Implications of compactness in metric spaces]
    Let $\met$ be a metric space.  If $A$ is a compact subset in $X$, then $A$ is closed and bounded in $X$.  
\end{thm}

In $\R^n$, we have the following, more general, results.  These are key results in characterising compact subsets of Euclidean space.

\begin{thm}[Heine-Borel]
    Every closed and bounded interval in $\R$ is compact.
\end{thm}

\begin{thm}[Compactness in $\R^n$]
    Let $A$ be a subset of $\R^n$.  Then $A$ is compact if and only if $A$ is closed and bounded.
\end{thm}

\subsection{Properties of compact sets} % (fold)
\label{sub:properties_of_compact_sets}
\begin{thm}
    A subset $A$ in $\R^n$ is compact if and only if every sequence in $A$ has a convergent subsequence with limit in $A$.
\end{thm}

\begin{defn}[Compactness is a topological property]
    Let $X$ and $Y$ be topological spaces, and let $f : X \rightarrow Y$ be a continuous mapping.  If a subset $A$ in $X$ is compact, then the image $f(A)$ is compact in $Y$.
    
    That is, continuous images of compact sets are compact.
\end{defn}

\begin{cor}
    The following are true in an arbitrary metric space.
    \begin{itemize}
        \item A continuous image of a compact subset is closed.
        \item A continuous image of a compact subset is bounded.
    \end{itemize}
\end{cor}

\begin{thm}
    Any closed subspace $A$ of a compact space $X$ is compact.
\end{thm}

\begin{proof}
    TO DO
\end{proof}
% subsection properties_of_compact_sets (end)
% section compactness (end)


\section{Applications to Continuous Functions $f : [a,b] \rightarrow \R$} % (fold)
\label{sec:applications_to_f_a_b_rightarrow_r_}
\begin{thm}[Intermediate Value Theorem]
    Let $f: [a,b] \rightarrow \R$ be a continuous function.  Let $K$ be a number lying between $f(a)$ and $f(b)$.  Then there exists a point $c \in [a,b]$ such that $f(c) = K$.
\end{thm}

\begin{proof}
    Using connectedness of $[a,b]$, we have that $f([a,b])$ is connected.  Thus, if $f(a) < K < f(b)$, then $K \in f([a,b])$.  Hence, there exists $c \in [a,b]$ such that $f(c) = K$.
\end{proof}
% section applications_to_f_a_b_rightarrow_r_ (end)


\section{Topological Spaces} % (fold)
\label{sec:topological_spaces}

A topological space is defined as follows.  Let $X$ be a non-empty set.  Then a family $\tau$ of subsets of $X$ is called a topology for $X$ if $\tau$ satisfies 
\begin{itemize}
    \item $\emptyset, X \in \tau$;
    \item The union of any subfamily of members of $\tau$ is in $\tau$;
    \item The intersection of any \textbf{finite} subfamily of members of $\tau$ is in $\tau$.
\end{itemize}

The pair $(X, \tau)$ is called a \textbf{topological space}, and the members of $\tau$ are called the \textbf{open sets} in $(X, \tau)$. 


\begin{defn}[Interior of a subset]
    Let $\topo$ be any topological space and $A \subseteq X$.  Then $a \in A$ is called an \textbf{interior point} of $A$ if there exists an open set $U$ containing $a$ such that $U \subseteq A$.  We denote by $\intr A$ the set of all interior points of $A$.
\end{defn}

\begin{thm}[Properties of the interior]
    Let $\topo$ be any topological space and let $A \subseteq X$.  Then $\intr A$ is the \textbf{largest} open subset contained in $A$.
\end{thm}

\begin{defn}[Closed subsets]
    Let $A$ be a subset of a topological space $\topo$.  Then $A$ is called a \textbf{closed set} in $X$ if the complement $X \backslash A$ is open in $X$, that is, if $X \backslash A \in \tau$.
\end{defn}

\begin{thm}
    In a topological space $\topo$,
    \begin{itemize}
        \item $\emptyset$ and $X$ are closed;
        \item The intersection of any collection of closed sets is closed;
        \item The union of any \textbf{finite} collection of closed sets are closed.
    \end{itemize}
\end{thm}

\begin{defn}[Limit point of a subset]
    Let $\topo$ be a topological space and let $A \subseteq X$.  Then a point $x \in X$ is called a \textbf{limit point} or \textbf{accumulation point} of $A$ if every open set $G$ containing $x$ contains a point of $A$ different from $x$, i.e.,\[
        G \in \tau, x \in G \Rightarrow (G \backslash \{x \}) \cap A \neq \emptyset
    \]
We denote by $A'$ the set of all limit points of $A$, and is called the derived set of $A$.
\end{defn}

\begin{thm}[Properties of closed sets]
    Let $\topo$ be a topological space and let $A \subseteq X$.  Then $A$ is closed if and only if $A' \subseteq A$.
\end{thm}


\begin{defn}[Closure of a subset]
    Let $\topo$ be a topological space and $A \subseteq X$.  Then the set consisting of $A$ together with all its limit points is called the \textbf{closure} of $A$, and is denoted by $\ol A$.  Thus,\[
        \ol A = A \cup A'
    \]
\end{defn}

\begin{prop}
    Let $\topo$ be a topological space and $A \subseteq X$.  Then \[
        \ol A = \{ x \in X \, | \, \text{for every open set $U$ containing $x$, $U \cap A \neq \emptyset$} \}.
    \]
\end{prop}

\begin{thm}
        Let $\topo$ be a topological space and $A \subseteq X$.  Then $\ol A$ is the smallest closed set containing $A$.
\end{thm}

\begin{defn}[Dense subset]
        Let $\topo$ be a topological space and $A \subseteq X$.  Then a subset $A$ of $X$ is said to be \textbf{dense} in $X$ if $\ol A = X$.
\end{defn}

\begin{defn}[Nowhere dense]
        Let $\topo$ be a topological space and $A \subseteq X$. Then $A$ is \textbf{nowhere dense} in $X$ if and only if the interior of the closure of $A$ is empty.  That is, $\intr (\ol A) = \emptyset$.  
        
        Alternatively, a set is nowhere dense if and only if $X \backslash \ol A$ is dense in $X$.
\end{defn}

\begin{defn}[Boundary of a subset]
    Let $\topo$ be a topological space and $A \subseteq X$.  Then the \textbf{boundary} of $A$, denoted by $\partial A$, is defined as \[
        \partial A = \ol A \cap \ol{X \backslash A}
    \]
\end{defn}

\begin{thm}[Characterisation of the boundary]
        Let $\topo$ be a topological space and $A \subseteq X$. Then \[
            \ol A = \intr A \cup \partial A
        \]
\end{thm}

\begin{defn}[Convergence in topological spaces]
    Let $\topo$ be a topological space.  Then a sequence $(x_n)$ of points in $X$ is said to \textbf{converge} to a point $x \in X$ if for any open set $U$ containing $x$, there exists a positive integer $N$ such that \[
        n > N \Rightarrow x_n \in U
    \]
    That is, if any open set $U$ containing $x$ contains almost all of the terms of the sequence.
\end{defn}

\begin{defn}[Induced or relative topology]
        Let $\topo$ be a topological space and $Y \subseteq X$. Let \[
            \tau_Y = \{ G \subseteq Y \, | \, G = U \cap Y \text{ for some $U \in \tau$} \}
        \]
    Then $\tau_Y$ is a topology for $Y$, called the \textbf{induced} or \textbf{relative topology} on $Y$ and the space $(Y, \tau_Y)$ is called a subspace of $\topo$
\end{defn}

\begin{defn}[Bases for a topology]
    Let $\topo$ be a topological space.  Then a subfamily $\mathcal{B}$ of $\tau$ is called a \textbf{base} for the topology if for every open set $U$ in $\tau$ is the union of members of $\mathcal{B}$.  Equivalently, $\mathcal{B} \subseteq \tau$ is a basis for $\tau$ if and only if for any point $a$ in an open set $U \in \tau$, there exists $V \in \mathcal{B}$ such that $a \in V \subseteq U$.
\end{defn}

\begin{thm}[Characterisation of a basis for a topology]
    A family of nonempty subsets of a set $X$ is a base for some topology $\tau$ on $X$ if and only if it satisfies the following properties.
    \begin{itemize}
        \item $X = \cup_{B \in \mathcal{B}} B$
        \item For any $B_1, B_2 \in \mathcal{B}$, $B_1 \cap B_2$ is the union of members of $\mathcal{B}$.  Equivalently, if $b \in B_1 \cap B_2$, then there exists $B_b \in \mathcal{B}$ such that $b \in B_b \subseteq B_1 \cap B_2$.
    \end{itemize}
\end{thm}

\begin{thm}[Continuity in terms of a basis]
    Let $(X, \tau_X)$ and $(Y, \tau_Y)$ be topological spaces and $f : X \rightarrow Y$ a mapping.  Let $\mathcal{B}_Y$ be a basis for $\tau_Y$.  Then $f$ is continuous if and only if for any $B \in \mathcal{B}_Y$, $f^{-1}(B)$ is open in $X$ - i.e., is in $\tau_X$.
\end{thm}


\begin{thm}[Product spaces]
    Let $(X, \tau_X)$ and $(Y, \tau_Y)$ be topological spaces.  Then the family $\mathcal{B}$ given by \[
        \mathcal{B} = \{ U \times V \, | \, U \in \tau_X, V \in \tau_Y \}
    \]
    is a base for a topology on $X \times Y$.
\end{thm}

\subsection{Compactness} % (fold)
\label{sub:compactness}

\begin{defn}[Compactness in terms of a basis]
    A topological space $\topo$ is compact if and only if there exists a base $\mathcal{B}$ for $\tau$ such that every open covering of $X$ be members of $\mathcal{B}$ has a finite subcovering. 
\end{defn}

\begin{thm}
    The product space of two compact topological spaces is compact.
\end{thm}

% subsection compactness (end)

\subsection{Connectedness} % (fold)
\label{sub:connectedness}

\begin{thm}
    Let $X$ and $Y $ be topological spaces and let $f : X \rightarrow Y$ be a homeomorphism.  Then $X$ is connected if and only if $Y$ is connected.    
\end{thm}

\begin{thm}
    Let $X$ be a topological space.  Let $\{ A_i \}$be a family of connected subsets in $X$ and suppose that for all $i,j$, $A_i \cap A_j \neq \emptyset$.  Then the union $A = \cup_i A_i$ is connected.
\end{thm}

% subsection connectedness (end)

% section topological_spaces (end)


\section{Separation Properties} % (fold)
\label{sec:spearation_properties}

\begin{defn}[$T_0$-spaces]
    A topological space $\topo$ is called a \textbf{$T_0$-space} if for any pair of distinct points $a,b$ of $X$, either there exists an open set $U$ containing $a$ and not $b$ or an open set $V$ containing $b$ and not $a$.
\end{defn}

\begin{defn}[$T_1$-spaces]
A topological space $\topo$ is called a \textbf{$T_1$-space} if for any pair of distinct points $a,b$ of $X$, there exists an open set $U $ in $X$ with $a \in U$ and $b \notin U$.
\end{defn}

Every $T_1$-space is $T_0$, but not the reverse.

\begin{thm}[Characterisation of $T_1$-spaces]
    AA topological space $\topo$ is called a \textbf{$T_1$-space} if and only if eery singleton set $\{ a \}$ of $X$ is closed (and so every finite subset of $X$ is closed).
\end{thm}
    
\begin{defn}[$T_2$-spaces or Hausdorff spaces]
    A topological space $\topo$ is called a \textbf{$T_2$-space} or a \textbf{Hausdorff space} if for any pair of distinct points $a,b$ of $X$, there are disjoint open sets $U$ and $V$ in $X$ such that $a \in U$ and $b \in V$.
\end{defn}

Every $T_2$-space is $T_1$, but not the reverse.

\begin{exmp}
    Any metric space is a $T_2$-space
\end{exmp}
    
\begin{thm}
    Every subspace of a $T_1$- or $T_2$-space is a $T_1$- or $T_2$-space. 
\end{thm}

\begin{thm}
    Every product space of a $T_1$- or $T_2$-space is a $T_1$- or $T_2$-space. 
\end{thm}

\subsection{Regular Spaces and $T_3$-spaces} % (fold)
\label{sub:regular_spaces}

\begin{defn}[Regular space]
    A topological space $\topo$ is called a \textbf{regular} space if for any closed set $F$ in $X$ and $a \in X \backslash F$, there are disjoint open sets $U$ and $V$ in $X$ such that $F \subseteq U$ and $a \in V$.  A regular $T_1$-space is called a $T_3$-space.
\end{defn}

\begin{thm}
    A topological space $X$ is regular if and only if for any point $a$ in $X$ and any open set $U$ containing $A$, there is an open set $W$ containing $a$ such that $\ol W \subseteq U$.

\end{thm}

Every $T_3$-space is $T_2$, but not the reverse.


% subsection regular_spaces (end)
\subsection{Normal Spaces and $T_4$-spaces} % (fold)
\label{sub:normal_spaces_and_t_4_spaces}

\begin{defn}[Normal spaces and $T_4$-spaces]
    A topological space $\topo$ is called a \textbf{normal space} if for any two disjoint closed sets $A$ and $B$ in $X$, there are disjoint open sets $U$ and $V$ in $X$ such that $A \subseteq U$ and $B \subseteq V$.  A normal $T_1$-space is called a $T_4$-space.
\end{defn}

Clearly every $T_4$-space is a $T_3$-space.  However, a normal space may not be a $T_1$-space or a regular space and a regular space may not be normal.

\begin{thm}
    Every metric space $\met$ is a $T_4$-space
\end{thm}

\begin{thm}
    Every compact Hausdorff space is normal.  Additionally, any compact subset $A$ of a Hausdorff space $X$ is closed.
\end{thm}
% subsection normal_spaces_and_t_4_spaces (end)

Our final theorem is \textbf{Urysohn's Lemma}.
\begin{thm}[Urysohn's Lemma]
    Let $X$ be a normal space.  Then, for any disjoint closed sets $A$ and $B$ in $X$, there exists a continuous function $f: X \rightarrow [0,1]$ such that $F(A) = \{0 \}$ and $F(B) = \{1 \}.$
\end{thm}

\begin{thm}[Tietze Extension Theorem]
    Let $X$ be a normal space, $A$ a closed subset of $X$, and $f : A \rightarrow \R$ continuous.  Then there is a continuous function $g: X \rightarrow R$ such that $g |_A = f$ - that is, $g$ restricted to $A$ is $f$. 
\end{thm}

% section spearation_properties (end)

\section{Hilbert Spaces} % (fold)
\label{sec:hilbert_spaces}
Let $V$ be a vector space over a field $\F$, where $\F$ is either $\R$ or $\Com$.

\begin{defn}[Inner product space]
    A function $\langle \cdot, \cdot \rangle: V \times V \rightarrow \F$ is called an inner product if 
    \begin{itemize}
        \item $\iprod{u}{v} = \ol{\iprod{u}{v}}$ for all $u,v \in V$.
        \item $\iprod{u}{u} \geq 0$ for all $u \in V$ with equality if and only if $u = 0$.
        \item $\iprod{\alpha u + \beta v}{w} = \alpha \iprod{u}{w} + \beta \iprod{v}{w}$ for all $u,v,w \in V$ and $\alpha, \beta \in \F$.
    \end{itemize}
    We say that $V$ equipped with $\iprod{\cdot}{\cdot}$ is an \textbf{inner product space}.
\end{defn}

\begin{defn}[Induced norm]
    If $V$ is an inner product space with $\iprod{\cdot}{\cdot}$, we define \[
        \| u \| := \sqrt{\iprod{u}{u}}
    \] for all $u \in E$.  The operation $\| \cdot \|$ is the \textbf{induced norm}.
\end{defn}

\begin{thm}[Cauchy-Schwarz inequality]
    Let $V$ be an inner product space.  The \[
        |\iprod{u}{v}| \leq \|u \| \| v \|
    \]
\end{thm}

\begin{defn}[Hilbert space]
    An inner product space which is complete with respect to the induced norm is called a \textbf{Hilbert space}.
\end{defn}

\begin{prop}[Continuity of the inner product]
    Let $V$ be an inner product space.  Then the inner product is continuous with respect to the induced norm.  
\end{prop}

\begin{prop}[Parallelogram identity]
    Let $V$ be an inner product space and $\| \cdot \|$ the induced norm.  Then \[
        \| u + v \|^2 + \| u - v \|^2 = 2 \|u \|^2 + 2 \| v \|^2
    \] for all $u,v \in E$
\end{prop}

\subsection{Projections and orthogonal complements} % (fold)
\label{sub:projections_and_orthogonal_complements}

\begin{defn}[Projection]
    Let $V$ be a normed space and $M$ a non-empty closed subset of $V$.  We define the \textbf{set of projections of $x$ onto $M$} by \[
        P_M(x) = \{ m \in M \, | \, \|x - m \| = d(x, M) \}
    \]
    This set is non-empty as $M$ is closed.
\end{defn}
\begin{defn}[Orthogonal complement]
    For an arbitrary non-empty subset $M$ of an inner product space $H$ we set \[
        M^{\perp} := \{ x \in H \, | \, \iprod{x}{m} = 0 \text{ for all $m \in M$} \}
    \]
    We call $M^\perp$ the orthogonal complement of $M$ in $H$.
\end{defn}

\begin{lem}[Properties of the orthogonal complement]
    Suppose $M$ is a non-empty subset of the inner product space $H$.  Then $M^\perp$ is a closed subspace of $H$ and $M^\perp = \ol M^\perp = (\text{span $M$})^\perp = (\text{span $\ol M$})\perp$
\end{lem}

\begin{thm}[Key properties of the orthogonal complement]
    Suppose that $M$ is a closed subspace of a Hilbert space $H$. Then 
    \begin{itemize}
        \item $H = M \oplus M^\perp$ 
    \end{itemize}
    
    \end{thm}

\begin{cor}[Dense subspace of a Hilbert space]
    A subspace $M$ of a Hilbert space $H$ is dense in $H$ if and only if $M^\perp = \{0 \}$.
\end{cor}

% subsection projections_and_orthogonal_complements (end)

\subsection{Orthogonal systems} % (fold)
\label{sub:orthogonal_systems}

\begin{defn}[Orthogonal systems]  
    Let $H$ be an inner product space with inner product $\iprod{\cdot}{\cdot}$ and induced norm $\| \cdot \|$.  Let $M \subset H$ be a non-empty subset.
    \begin{itemize}
        \item $M$ is called an \textbf{orthogonal system} if $\iprod{u}{v} = 0$ for all $u,v \in M$ with $u \neq v$.
        \item $M$ is called an \textbf{orthonomal system} if it is orthogonal and $\| u \| = 1$.
        \item $M$ is called a \textbf{complete orthonormal sytem} or \textbf{orthonormal basis} of $H$ if it is an orthonormal system and $\ol{\text{span $M$}} = H$.
    \end{itemize}

\end{defn}

\begin{thm}[Pythagoras's Theorem]
    Suppose that $H$ is an inner product space and $M$ an orthogonal system in $H$.  Then the following assertions are true:
    \begin{itemize}
        \item $M \backslash \{0\}$ is linearly independent.
        \item If $(x_n)$ is a sequence in $M$ with distinct terms and $H$ is complete, then $\sum x_k$ converges if and only if $\sum \| x_k \|^2$ converges.  In that case, \[
            \left\| \sum x_k \right\|^2 = \sum \|x_k \|^2
        \]
    \end{itemize}
\end{thm}

\begin{thm}[Bessel's Inequality]
    Let $H$ be an inner product space and $M$ an orthonormal system in $H$.  Then \[
        \sum_{m \in M} | \iprod{x}{m}|^2 \leq \| x \|^2
    \]
    for all $x \in H$.  Moreover, the set $\{ m \in M \, | \, \iprod{x}{m} \neq 0 \}$ is at most countable for all $x \in H$.
\end{thm}

% subsection orthogonal_systems (end)
% section hilbert_spaces (end)





\end{document}