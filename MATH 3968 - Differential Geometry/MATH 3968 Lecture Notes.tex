%
%  untitled
%
%  Created by Andrew Tulloch on 2009-10-30.
%  Copyright (c) 2009 __MyCompanyName__. All rights reserved.
%%!TEX TS-program = xelatex
%!TEX encoding = UTF-8 Unicode

\documentclass[10pt, oneside, reqno]{amsart}
\usepackage{geometry, setspace, graphicx, enumerate}
\onehalfspacing                 
\usepackage{fontspec,xltxtra,xunicode}
\defaultfontfeatures{Mapping=tex-text}





\newcommand{\deriv}[1]{\frac{\partial}{\partial u^{#1}}}
\newcommand{\christ}[3]{\ensuremath{\Gamma^{#1}_{#2#3}}}
\newcommand{\R}{\mathbb{R}}

		\newcommand{\cov}{\text{cov}}
			% AMS Theorems
			\theoremstyle{plain}% default 
			\newtheorem{thm}{Theorem}[section] 
			\newtheorem{lem}[thm]{Lemma} 
			\newtheorem{prop}[thm]{Proposition} 
			\newtheorem*{cor}{Corollary} 

			\theoremstyle{definition} 
				\newtheorem{defn}[thm]{Definition}
				\newtheorem{conj}[thm]{Conjecture}
				\newtheorem{exmp}[thm]{Example}

			\theoremstyle{remark} 
				\newtheorem*{rem}{Remark} 
				\newtheorem*{note}{Note} 
				\newtheorem{case}{Case} 
		
\title{MATH 3968 Notes}								% Document Title
\author{Andrew Tulloch}
%\date{}                                           % Activate to display a given date or no date


\begin{document}
\maketitle

\section{Curves in $\mathbb{R}^N$}
\begin{defn}
	A curve is said to be \emph{regular} if $\alpha'(t) \neq 0$ for all $t$
\end{defn}

\begin{defn}
	Let $\alpha$ be a curve parametrized by arc length.  The number $| \alpha''(s)| = k(s)$ is called the \emph{curvature} of $\alpha$ at $s$.
\end{defn}

\begin{defn}[Frenet equations]
	\begin{align*}
		\mathbf{t'} = k \mathbf{n} \\
		\mathbf{n'} = -\mathbf{k}t + \tau \mathbf{b}\\
		\mathbf{b'} = - \tau \mathbf{n}\\
	\end{align*}	
\end{defn}

\begin{defn}
	Total curvature of a unit speed curve $\alpha(s)$ from $a$ to $b$ is defined as \[
		\int_a^b k ds = \theta(b)-\theta(a)
	\]
\end{defn}

\begin{defn}
	The winding number $k$ is defined as the integer such that, for a closed curve, the total curvature \[
		\int_a^b k ds = 2 \pi k
	\]
\end{defn}

\begin{thm}[Rotation Theorem]
	For a piecewise smooth simple closed curve, traversed in an anti-clockwise direction around a bounded region of the plane, the winding number is 1
\end{thm}

\section{General Analysis}
\begin{defn}[Homeomorphism]
	A map is a \emph{homeomorphism} is it is continuous and has a continuous inverse.
\end{defn}

\begin{defn}[Diffeomorphism] 
	A map is a \emph{diffeomorphism} is it is smooth and has a smooth inverse
\end{defn}

\begin{defn}[Differential of a smooth map]
	The differential of a smooth map $\varphi: \R^N \rightarrow \R^M$ is defined as follows.  Let $w \in R^M$, and let $\alpha$ be a differentiable curve such that $\alpha(0) = p$, $\alpha'(0) = w$.  Then the composition $\beta = \varphi \circ \alpha$ is differentiable, and we have \[
		d \varphi_p (w) = \beta'(0) = (\varphi \circ \alpha)'(0)
	\]
	
\end{defn}

\begin{thm}[Inverse function theorem]
	Let  $\phi : U \subset \R^N \rightarrow \R^M$ be a differentiable mapping and suppose that the differential $d \phi$ is an isomorphism at $p \in U$.  Then there exists a neighbourhood $V \subset U$ and a neighbourhood $W$ of $F(p)$ such that $\phi$ is a diffeomorphism. 
\end{thm}

\begin{thm}[Implicit function theorem]
	%% TODO Find this
	
\end{thm}

\begin{defn}
	A map $\phi$ is \emph{regular} at a point $p$ if the matrix of $d\phi$ has full rank at $p$.  A critical point has that the matrix of $d\phi$ does not have a full rank at $p$.
\end{defn}

\begin{defn}[Implicitly defined surfaces]
	Let $f: \R^3 \rightarrow \R$ be a smooth function.  Let $p$ be a point in $\R^3$ with $f(p) =0$ and $f'(p) \neq 0$.  Then there exists a neighbourhood of the point $a$ such that there exists a smooth parameterisation $\phi : U \rightarrow \R^3$ of the set of solutions to $f(x) = 0$.
\end{defn}

\begin{rem}
	This allows us to define surfaces implicitly as the set of solutions to some function.  E.g. the sphere in $\R^3$, the sphere $S^2$ is defined as the set of solutions to \[
		f(x,y,z) = x^2 +y^2 + z^2 - 1 = 0
	\]
\end{rem}

\section{Surfaces in $\R^3$}



\begin{defn}
	A \emph{regular surface} is defined intuitively that for every point $p \in S$, there exists a diffeomorphism between some open set $U \in \R^2$ and a neighbourhood $V$ of that point $p \in S$
\end{defn}

\begin{defn}
	Let $\phi: U \rightarrow R^3$ be a parametrization of a surface S.  The tangent space $T_p S$ to the surface at a point $p \in U$ is the image of the mapping \[
		d\phi_p : \R^2 \rightarrow \R^3
	\]
\end{defn}
\begin{thm}[Surface as a graph]
	Let $S \subset \R^3$ be a regular surface and $p \in S$.  Then there exists a neighbourhood $V$ of $p$ in $S$ such that $V$ is the graph of a differentiable function which has one of the following forms: $z = f(x,y), y = g(z,x), x = h(z,y)$.
\end{thm}


\begin{defn}[Surface of revolution]
	Let $C(v) = (\alpha(v),0,\beta(v))$ be a curve in the $xz$ plane.  Then we can parameterise the surface formed by revolving the curve about the $z$ axis by \[
		\phi(u,v) = (\alpha(v)\cos u, \alpha(v) \sin u,\beta v )
	\]
\end{defn}
\begin{defn}[First fundamental form]
	Let $S$ be a surface parameterised by $\phi : (u^1,u^2) \subset \R^2 \rightarrow \R^3$.  Write $\mathbf{E_i} = \frac{\partial \phi}{\partial u^i}$.  Then we can define the matrix $g$ of the second fundamental form $I$  as $g_{ij} = \langle E_i,E_j \rangle$
\end{defn}

\begin{defn}
	The arc length of a parameterised curve $\alpha(t) = \phi(u^1(t),u^2(t))$ is given by \[
		s(t) = \int_0^t |\alpha'(t)| dt = \int_0^t \sqrt{\sum g_{ij} u'^i u'^j }
			\]
\end{defn}

\begin{defn}[Area of a regular surface]
	Let $R \subset S$ be a bounded region of a regular surface, contained in the coordinate neighbourhood of the parameterization $\phi: U \subset \R^2 \rightarrow \R^3$.  The positive number \[
		\iint_Q |\mathbf{E_1} \times \mathbf{E_2} | du^1 du^2 = A(R), \qquad Q = \phi^{-1}(R) 
	\]
\end{defn}

\begin{defn}[Orientation]
	A regular surface is \emph{orientable} if it is possible to cover it with a family of coordinate neighbourhoods in such a way that if a point $p \in S$ belongs to two neighbourhoods of this family, then the change of coordinate matrix has a positive determinant at $p$.  If such a choice is not possible, then the surface is called \emph{non-orientable}.
\end{defn}

\begin{defn}[Normal vector]
	We define the \emph{normal vector} \textbf{N} to a regular surface $S$ as \[
		\mathbf{N}(p) = \frac{E_1 \times E_2}{|E_1 \times E_2|}(p)
	\]
\end{defn}

\begin{defn}[Gauss Map]
	Let $S \subset \R^3$ be a regular surface with an orientation $\mathbf{N}$.  The map $\mathbf{N}: S \rightarrow \R^3$ takes its values in the unit sphere $S^2$, and this map $\mathbf{N}: S \rightarrow S^2$ is called the \emph{Gauss map}.  
\end{defn}

\begin{prop}
	The differential $dN_p : T_p(S) \rightarrow T_p(S)$ is a self adjoint linear map - \emph{i.e.} \[
		\langle dN_p(w),v \rangle = \langle w,dN_p(v) \rangle
	\]
\end{prop}

\begin{defn}
	The matrix of the second fundamental form is given by \[ II_{ij} = h_{ij} = -\langle \mathbf{N_{u_i}},\mathbf{E_j} \rangle = -\langle  \mathbf{dN(E_i)},\mathbf{E_j} \rangle = \langle \mathbf{N}, \mathbf{E_{ij}} \rangle \]

We have \[
	II_p(X) = - \langle dN_p(X),X \rangle
\]
\end{defn}

\begin{defn}[Normal curvature]
	The \emph{normal curvature} of a space curve $k_n$ is defined as $k_n = k cos \theta$, where $\theta$ is the angle between \textbf{N} and \textbf{n}.
\end{defn}

\begin{defn}
	The \emph{principle curvatures} are the eigenvalues of $-dN_p$.     
\end{defn}

\begin{thm}
	If the normal curvatures $	II_p(X) $ with $|X| = 1$ are not all equal, then there is an orthonormal basis $\mathbf{e_1}, \mathbf{e_2}$ of $T_p S$ such that \[
			II_p(\cos \theta \mathbf{e_1} + \sin \theta \mathbf{e_2} = k_1 \cos^2 \theta + k_2 \sin^2 \theta
	\]
	where $k_1,k_2$ are the principle curvatures as defined above.
\end{thm}

\begin{defn}
	A point $p \in S$ where $k_1 = k_2$ is called an umbilical point
\end{defn}

\begin{defn}
	A curve $\alpha(t)$ is \emph{asymptotic} if its normal curvature is zero for all $t$.
\end{defn}

\begin{defn}
	A curve is a \emph{line of curvature} if $\alpha'(t)$ is an eigenvector of $-dN_{\alpha(t)}$ for all $t$
\end{defn}
\begin{defn}
	The \emph{Gauss curvature} of an oriented surface at a point $p$ is \[
		K(p) = \det(-dN_p) = k_1(p) k_2(p)
	\]
	
	The \emph{mean curvature} of $S$ at $p$ is \[
		H(p) = \frac{1}{2} \text{trace} (-dN_p)
	\]
\end{defn}


\begin{thm}
	\[
	-(dN_p)^T = (II)(I)^{-1}	
	\]
	\[
		K = \frac{h_{11}h_{22}-{h^2_{12}}}{g_{11}g_{22}-g^2_{12}}
	\]
	\[
		H = \frac{h_{11}g_{22}-2h_{12}g_{12}+h_{22}g_{11}}{2 ({g_{11}g_{22}-g^2_{12})}}
	\]
\end{thm}

\begin{defn}[Minimal surface]
	A surface is a \emph{minimal surface} if its mean curvature is identically zero. \[
		H \equiv 0
	\]
\end{defn}

\begin{defn}
	A coordinate chart $\phi: U \subset \R^2 \rightarrow \Sigma \subset \R^3$ is called \emph{isothermal} if $g_{11} = g_{22}, g_{12} = g_{22} = 0$.  This is equivalent to saying $(I)$ is a multiple of the identity.
\end{defn}

\begin{thm}
	If $\phi$ is isothermal, then $\phi(U) = \Sigma$ is a minimal surface if and only if $\nabla^2 \phi = \mathbf{0}$
\end{thm}

\begin{defn}[Isometry]
	A diffeomorphism \[
		\psi: \S_1 \rightarrow S_2 
	\]
	is an isometry if it preserves the inner product, i.e. for all $p \in S_1$ and $X,Y \in T_P S_1$ \[
		\langle X,Y \rangle_p = \langle d \psi_p(X),d \psi_p(Y) \rangle_{\psi(p)}
	\]
	
	This is equivalent to \[
		I_p(X) = I_{\psi(p)} (d\psi_p(X)
	\]
	If for each $p \in S_1$, there is an open neighbourhood $U_1$ of p and a map $\psi : U_1 \rightarrow S_2$, then we say that $S_1$ is \emph{locally isometric} to $S_2$.
\end{defn}

\begin{thm}
	If two surfaces $\psi: U \rightarrow S_1$ and $\varphi: U \rightarrow S_2 $  have the same coefficients for $g_{ij}$, then $S_1$ and $S_2$ are locally isometric.
\end{thm}
\begin{defn}[Christoffel symbols]
	Define the \emph{Christoffel symbols} $\christ{i}{j}{k}$ by 
	\begin{align*}
		\phi_{u_1 u_1} & = \christ{1}{1}{1} \phi_{u_1} + \christ{2}{1}{1} \phi_{u_2} + L_1 \mathbf{N} \\
		\phi_{u_1 u_2} & = \christ{1}{1}{2} \phi_{u_1} + \christ{2}{1}{2} \phi_{u_2} + L_2 \mathbf{N} \\
		\phi_{u_2 u_1} & = \christ{1}{2}{1} \phi_{u_1} + \christ{2}{2}{1} \phi_{u_2} + \hat{L}_2 \mathbf{N}\\
		\phi_{u_2 u_2} & = \christ{1}{2}{2} \phi_{u_1} + \christ{2}{2}{2} \phi_{u_2} + L_3 \mathbf{N}\\
	\end{align*}
	
	Taking inner products with $\mathbf{E_i}$ gives
	
	\begin{align*}
		\christ{1}{1}{1}g_{11} + \christ{2}{1}{1}g_{12} &= \frac{1}{2}(g_{11})_{u_1} \\
			\christ{1}{1}{1}g_{12} + \christ{2}{1}{1}g_{22} &= (g_{12})_{u_1} - \frac{1}{2}(g_{11})_{u_2}\\
				\christ{1}{1}{2}g_{11} + \christ{2}{1}{2}g_{12} &= \frac{1}{2}(g_{11})_{u_2}\\
					\christ{1}{1}{2}g_{12} + \christ{2}{1}{2}g_{22} &= \frac{1}{2}(g_{22})_{u_1} \\
						\christ{1}{2}{2}g_{11} + \christ{2}{2}{2}g_{12} &= (g_{12})_{u_2} - \frac{1}{2}(g_{22})_{u_1}\\
							\christ{1}{2}{2}g_{12} + \christ{2}{2}{2}g_{22} &= \frac{1}{2}(g_{22})_{u_2}
	\end{align*}
\end{defn}

\begin{thm}[Gauss' Theorem]
\[	g_{11}K = (\christ{2}{1}{1})_{u_2} - (\christ{2}{1}{2})_{u_1} + \christ{2}{1}{1} \christ{2}{2}{2} + \christ{1}{1}{1}\christ{2}{1}{2} - \christ{1}{1}{2}\christ{2}{1}{1} - \christ{2}{1}{2}\christ{2}{1}{2}     \]
\end{thm}

\begin{thm}[Mainardi-Codazzi equations]
	\begin{align*}
		(h_{11})_{u_2} - (h_{12})_{u_1} = h_{11} \christ{1}{1}{2} + h_{12}(\christ{2}{1}{2}-\christ{1}{1}{1}) - h_{22}\christ{2}{1}{1} \\
			(h_{12})_{u_2} - (h_{22})_{u_1} = h_{11} \christ{1}{2}{2} + h_{12}(\christ{2}{2}{2}-\christ{1}{1}{2}) - h_{22}\christ{2}{1}{2} 
	\end{align*}
\end{thm}

\begin{thm}[Christoffel symbols in terms of the metric]
	\[
		\christ{l}{i}{k} = \frac{1}{2}g^{jl}\left( g_{ij,k} + g_{jk,i} - g_{ki,j} \right)
	\]
\end{thm}

\begin{defn}
	Let $w$ be a smooth vector field in a neighbourhood $U$ of $p \in S$.  Let $\alpha \subset S$ be a smooth curve with $\alpha(0) = p, \alpha'(0) = v$.  Define \[
		\frac{Dw}{dt}(0) = D_v w(p)
	\]
	to be the projection of $\frac{dw}{dt} = dw_v(p)$ to $T_p S$, and call it the covariant derivative at $p$ of the vector field $w$ in the direction of $v$.
\end{defn}


\begin{defn}
	Let the vector field $w = a^i \mathbf{E_i}$ and $v = b^j \mathbf{E_j}$.  Then we have \[
		\nabla w (v) = \nabla_v(w) = \frac{Dw}{dt} = \frac{da^i}{dt}\mathbf{E_i} + a^i b^j \christ{k}{i}{j} \mathbf{E_k}
	\]
\end{defn}

\begin{defn}[Parallel transport]
	A vector field $w$ along a curve $\alpha \subset S$ is called \emph{parallel} if \[
		\nabla_{\alpha'(t)}w = 0
	\]
	for all $t$.
\end{defn}

\begin{defn}
	A nonconstant parameterised curve $\alpha$ is a parameterised geodesic if \[
	\nabla_{\alpha'(t)} \alpha'(t) = 0	
	\]
	i.e. if its tangent vector field is parallel along the curve.
\end{defn}

\begin{defn}
	Let $w$ be a smooth field of unit vectors along a curve $\alpha$ where $\alpha$ lies in an oriented surface.  Then define the \emph{algebraic value of the covariant derivative} $[	\nabla_{\alpha'(t)}w]$ by \[
			\nabla_{\alpha'(t)}w = [	\nabla_{\alpha'(t)}w] \mathbf{N} \times w(t)
	\]
\end{defn}

\begin{defn}
		Let $\alpha$ be a curve parameterised by arc length.  The algebraic value of  $	\nabla_{\alpha'(t)} \alpha'(t)$ is called the \emph{geodesic curvature of $\alpha$} and is denoted by $k_g$.
\end{defn}

\begin{rem}
	Up to sign, $	\nabla_{\alpha'(t)} \alpha'(t)$ is the tangential projection of $\alpha''(s)$, and so the geodesic curvature is up to sign the tangential component of $a''(s)$.
\end{rem}

\begin{thm}
	For a curve $\alpha$ in an oriented surface $S$, we have \[
		k^2 = k_n^2 + k_g^2
	\]
\end{thm}

\begin{rem}
	Thus, $\alpha$ is a geodesic if and only if its geodesic curvature is identically zero.
\end{rem}

\begin{thm}
	Let $\gamma(t) = \phi(u^1(t), u^2(t))$.  Then $\gamma'(t) = (u^i)'\mathbf{E_i}$.  Then we have $\gamma$ is a geodesic if and only if 
	\begin{align*}
		0 &= 	\nabla_{\gamma'(t)} \gamma'(t) \\
		&= \nabla_{\gamma'(t)} (u^j)' \mathbf{E_j} \\
		&= (u^j)'' \mathbf{E_j} + (u^j)' \nabla_{(u^i)' \mathbf{E_i}} \mathbf{E_j} \\
		&=  [(u^k)'' + (u^j)' (u^i)' \christ{k}{i}{j}] \mathbf{E_k}
	\end{align*}
	if and only if \[
		(u^k)'' + (u^j)' (u^i)' \christ{k}{i}{j} = 0
	\] for $k = 1,2$
\end{thm}

\begin{thm}
A point and a tangent vector to that point uniquely determine a geodesic in a sufficiently small neighbourhood of that point.
\end{thm}

\begin{thm}
	For each $p \in \Sigma$, there is some neighbourhood $U$ of $p$ such that for any $q \in U$, the geodesic from $p$ to $q$ has length less than or equal to the length of any other path from $p$ to $q$.
\end{thm}

\begin{thm}
	Let $v$ and $w$ be smooth \emph{unit} vector fields along a curve $\alpha \subset S$.  Then \[[	\nabla_{\alpha'(t)} w] - [	\nabla_{\alpha'(t)} v] = \frac{d \theta}{dt} \]
	where $\theta$ is a smooth choice of angle from $v$ to $w$.
\end{thm}

\subsection{The Gauss Bonnet Theorem}
We first prove a local version, and then the full global theorem.

The depth of this (global) theorem lies in its topological content rather than in its geometric content, and its most striking feature is that it relates the two.

In fact, a corollary is

\begin{cor}
	For an orientable compact surface $S$, \[ \iint_S K d \sigma = 2 \pi \chi(S) \]
\end{cor}

\begin{thm}[Theorem of Turning Tangents]
	Let $S$ be an oriented surface, and $\phi : U \rightarrow S$ a coordinate chart.  Let $\alpha$ be a simple, closed, parameterised curve, regular in each interval $[ t_i, t_{i+1}]$. 
	
	Choose a smooth angle function $\theta_i$ on each $[t_i, t_{i+1}]$ which measures the angle from $\mathbf{E_1}$ to $\alpha'$.  Denote by $\gamma_i$ the external angle at $t_i$.  Then \[
		\sum_{i=0}^k (\theta_i(t_{i+1})- \theta_i(t_i)) + \sum_{i = 0}^k \gamma_i = \pm 2 \pi
	\]
	and the above sum is $2 \pi$ if $\alpha $ is a positively oriented curve.
\end{thm}

\begin{thm}[Local Gauss-Bonnet Theorem]
	Let $S$ be an oriented surface, $U \subset \R^2$ be homeomorphic to an open disk and $\phi: U \rightarrow S$ be a coordinate chart compatible with the orientation, such that $g_{12} = 0$.
	
	Let $R \subset \phi(U)$ be a simple region and $\alpha$ a positively oriented piecewise regular curve whose trace equals $\partial R$. Let $\alpha$ be parameterised by arc length, with vertices at $s_0, s_1, \dots, s_k$.
	
	Then \[
		\sum_{i=0}^k \int_{s_i}^{s_{i+1}} k_g(s) ds + \iint_R K d \sigma + \sum_{i = 0}^k \gamma_i = 2 \pi
	\]
\end{thm}

\begin{cor}
	Let $R \subset \phi(U)$ be a simple region with piecewise regular boundary, and $\alpha: [0,l] \rightarrow S$ be an arc length parametrisation of $\partial R$.   
	Take a unit vector $w_0 \in T_{\alpha(0)}S$ and let $w$ be the vector field along $\alpha$ generated by the parallel transport of $w_0$.  Then we have 
	\[ \Delta \theta = \theta(l) - \theta(0) = \iint_R K d \sigma \]
	where $\theta$ is a continuous choice of angle from $\mathbf{E_1}$ to $w$.
\end{cor}


\begin{defn}
	The \emph{Euler characteristic} $ \chi(S)$ of a surface $S$ is given by $\chi(S) = V - E + F$, where $V,E,F$ are the number of vertices, edges and faces of any triangulation of $S$.
\end{defn}

\begin{thm}
	If two surfaces $S_1$ and $S_2$ are homeomorphic, then $\chi(S_1) = \chi(S_2)$.
\end{thm}

\begin{thm}
	\[\chi(S) = 2 - 2g \]
	where $g$ is the number of holes of $S$, called the \emph{genus} of $S$.
\end{thm}

\begin{thm}[Global Gauss-Bonnet]
	Let $R \subset S$ be a regular region of an oriented surface with positively oriented boundary curves $C_i$. Write $\gamma_j$ for the external angles of the curves $C_i$.  THen \[
		\sum_{i} \int_{C_i} k_g(s) ds + \iint_R K d \sigma + \sum_j \gamma_j = 2 \pi \chi(S)
	\]
\end{thm}


\begin{defn}
	A point $p$ is said to be a \emph{singular point} of the smooth vector field $v$ on $S$ if \[
		v(p) = 0
	\]
	
	Take a positively oriented simple closed curve $\alpha$ enclosing an isolated singular point of a vector field $v$.  Let $\theta_v$ be a smooth choice of angle from $\mathbf{E_1}$ to $v$.  THen \[
		\theta_v(l) - \theta_v(0) = 2 \pi I
	\]
	for some \textbf{integer} $I$.  We call $I$ the \emph{index} of $v$ at $p$.
\end{defn}

\begin{thm}[Poincar\`{e}-Hopf Theorem]
	Let $v$ be a smooth vector field on $S$ with isolated singular points $p_1,\dots, p_n$.  Then \[
		\sum_{i=1}^n I_i = \chi(S)
	\]
\end{thm}

\begin{cor}[Hairy-Ball theorem]
	If $S$ is not homeomorphic to a torus, then any smooth vector field on $S$ must vanish somewhere.
\end{cor}

\begin{thm}[Morse's Theorem]
	Let $f: S \rightarrow \R$ be a smooth function on a compact oriented surface $S$ such that all critical points ($df_p = 0$) are non-degenerate ($\det A(p) \neq 0$, where $A$ is the \emph{Hessian} matrix of $f$).

Let
\begin{itemize}

	\item 	$M$ = number of local maxima, 

	\item 	$m$ = number of local minima, 

	\item 	$s$ = number of saddle points.

\end{itemize}
	
	Then \[
		M - s + m = \chi(S)
	\]
	This is independent of the function $f$, and depends only upon the topology of $S$.
\end{thm}

\section{Abstract Manifolds}

\begin{defn}[Abstract manifolds]
	
\end{defn}
An abstract surface is a set $\Sigma$ together with a family of maps \[
	\phi_\alpha: U_\alpha \rightarrow \Sigma
\] of open sets $U_\alpha \subset \R^2$ into $\Sigma$ so that

\begin{itemize}

	\item $\cup_\alpha \phi_\alpha(U_\alpha) = \Sigma$


	\item Whenever $\alpha, \beta$ are such that $W_{\alpha \beta} = \phi_\alpha (U_\alpha) \cap \phi_\beta (U_\beta) \neq \emptyset$, then $\phi^{-1}_\alpha (W_{\alpha \beta}), \phi^{-1}_\beta (W)_{\alpha \beta} \subset \R^2$ are open and
  \[
 	\phi^{-1}_\beta \circ \phi_\alpha : \phi^{-1}_\alpha (W_{\alpha \beta}) \rightarrow \phi^{-1}_\beta (W_{\alpha \beta})
 \]

\end{itemize}

\begin{defn}
	A function $f: M \rightarrow \R$ is \emph{smooth} at $p \in M$ if for all coordinate charts $\phi: U \rightarrow M$ about $p$, \[
		f \circ \phi : U \subset \R^N \rightarrow R
	\] is smooth
	
	A map $f : M_1 \rightarrow M_2$ between smooth manifolds is \emph{smooth} at $p \in M_1$ if for all coordinate charts $\phi : U \rightarrow M_1$ about $p$, and coordinate chart $\psi : V \rightarrow M_2$\[
		\psi^{-1} \circ f \circ \phi : U \subset \R^{N_1} \rightarrow \R^{N_2}
	\]
	It is called smooth if it is smooth at every point.

\end{defn}

\begin{defn}
	Two manifolds $M_1$ and $M_2$ are \emph{diffeomorphic} if there is a smooth map $f: M_1 \rightarrow M_2$ that has a smooth inverse.
	
	A necessary condition is that the manifolds have the same dimension.
\end{defn}

\begin{defn}
	A smooth parameterised curve in a (smooth) manifold $M$ is a smooth map $\alpha : (a,b) \rightarrow M$
\end{defn}

\begin{defn}[Tangent vectors as equivalence classes of curves]
	A \emph{tangent vector} $v$ to the smooth manifold $M$ at $p \in M$ is an equivalence class $[\alpha]$ of curve $\alpha: (-\epsilon, \epsilon) \rightarrow M$ with $\alpha(0) = p$. 
	
	Fix a coordinate chart $\phi: U \rightarrow M$ around $p$.  If $\alpha_1, \alpha_2$ satisfy $\alpha_1(0) = \alpha_2(0) = p$, then we define \[
		\alpha_1 \sim \alpha_2 \Leftrightarrow (\phi^{-1} \circ \alpha_1)'(0) = (\phi^{-1} \circ \alpha_2)'(0)
	\]
	
	This is independent of the choice of $\phi$
\end{defn}

\begin{defn}[Derivation]
	A derivation is a map from the real vector space of smooth function on $M$ to $\R$, which is linear and satisfies the product rule at $p$, \emph{i.e.} \[
		D(fg) = fD(g) + gD(f)
 	\]
\end{defn}

\begin{thm}
	THe set of all derivations at $p \in M^N$ forms an $n$-dimensional vector space, with basis \[
		\frac{\partial}{\partial u^i} : \phi \mapsto \frac{\partial \phi(p)}{\partial u^i}
	\]
	where \[
		\phi :(u^1,\dots, u^n) \mapsto \phi(u^1,\dots,u^n)
	\]
\end{thm}

\begin{defn}[Tangent vector as an operator that acts on functions]
	There is a one-to-one correspondence between equivalence classes of curves through $p$ and derivations at $p$.  We can thus alternatively define a tangent vector at $p$ to be a derivation at $p$.  
\end{defn}

\begin{defn}
	A smooth manifold $M$ is orientable if it has an atlas (i.e. may be covered with coordinate neighbourhoods) so that whenever $p ∈\in M$ lies in the image of two local parameterisations $\phi$ and $\psi$, the change of basis matrix at $p$ has positive determinant.
	
	If this is possible, then the choice of such an atlas is called an orientation of $M$, and we say that $M$ is oriented.
	
	An orientation of a smooth manifold M can be thought of as an orientation of every tangent plane that “varies smoothly”.
\end{defn}

\begin{defn}[Tangent Bundle]
Define \[
	TM = \{ (p,v): p ]in M, v \in T_p M \}
\]
\end{defn}

\begin{defn}
	A \emph{smooth vector field} on the smooth manifold $M$ is a smooth map $v : M \rightarrow TM$ such that $v_p \in T_pM$ for each $p \in M$.
\end{defn}

\begin{defn}
	Let $v,w$ be smooth vector fields on $M$.  A \emph{covariant derivative} on a smooth manifold $M$ is a map \[
		\nabla : (v,w) \mapsto \nabla_v w
 	\] 
such that 
\begin{itemize}


	\item $\nabla_v$ is linear in $v$ \emph{i.e.} $\nabla_{fu + gv}(w) = f \nabla_u(w) + g \nabla_v(w)$


	\item Each $\nabla_v$ satisfies linearity over $\R$: $\nabla_v(u +w) = \nabla_v(u) + \nabla_v(w)$


	\item Each $\nabla_v$ satisfies the product rule \[
 	\nabla_v (fu) = f \nabla_v(u) + v(f) u
 \]

\end{itemize}
\end{defn}

\begin{defn}[Levi-Civita connection]
	On a Riemannian manifold, there is a unique covariant derivative satisfying the following two conditions:
	\[
		\nabla_{\frac{\partial}{\partial u^i}} \frac{\partial}{\partial u^j} = \nabla_{\frac{\partial}{\partial u^j}} \frac{\partial}{\partial u^i} 
	\]
	and viewing $\frac{\partial}{\partial u^i}$ as a derivation,\[
		\deriv{i} (\langle \deriv{j},\deriv{k}\rangle) = \langle \nabla_{\deriv{i}} \deriv{j},\deriv{k}\rangle + \langle \deriv{j},\nabla_{\deriv{i}} \deriv{k} \rangle
	\]
	
\end{defn}

\begin{defn}[Christoffel symbols on an abstract manifold]
	Define the Christoffel symbols \christ{k}{i}{j} with respect to a local coordinate chart by \[
		\nabla_\deriv{i} \deriv{j} = \christ{k}{i}{j} \deriv{k}
	\]
	We have \[
		\christ{k}{i}{j} = \frac{1}{2}g^{kl}\left( g_{il,j} + g_{jl,i} - g_{ij,l} \right)
		\] which defines a unique covariant derivative.
\end{defn}

\begin{defn}[Geodesic]
	A geodesic is a curve $\alpha$ satisfying 
	\[ \nabla_\alpha' \alpha' = 0 \]
	
	This gives us the equations \[
		(u^k)'' + \christ{k}{i}{j}(u^i)'(u^j)' = 0
	\] for $k = 1,\dots,n$
\end{defn}


\begin{defn}
	The \emph{hyperbolic plane} is defined to be the upper half plane with the metric (writing $u^1 = x, u^2 = y$) \[
		g_{11} = \frac{1}{y^2}, g_{12} = 0, g_{22} = \frac{1}{y^2}
 	\]
\end{defn}


\end{document}