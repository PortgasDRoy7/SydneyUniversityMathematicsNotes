%
%  untitled
%
%  Created by Andrew Tulloch on 2009-11-04.
%  Copyright (c) 2009 __MyCompanyName__. All rights reserved.
%!TEX TS-program = xelatex
%!TEX encoding = UTF-8 Unicode

\documentclass[10pt, oneside, reqno]{amsart}
\usepackage{geometry, setspace, graphicx, enumerate}
\onehalfspacing                 
\usepackage{fontspec,xltxtra,xunicode}
\defaultfontfeatures{Mapping=tex-text}

%\setromanfont[Mapping=tex-text,Contextuals= 
%{NoWordInitial,NoWordFinal,NoLineInitial,NoLineFinal}]{Hoefler Text}
%\setsansfont[Scale=MatchLowercase,Mapping=tex-text]{Hoefler Text}
%\setmonofont[Scale=MatchLowercase]{Andale Mono}


	% AMS Theorems
	\theoremstyle{plain}% default 
	\newtheorem{thm}{Theorem}[section] 
	\newtheorem{lem}[thm]{Lemma} 
	\newtheorem{prop}[thm]{Proposition} 

	\theoremstyle{definition} 
		\newtheorem{defn}[thm]{Definition}
		\newtheorem{conj}[thm]{Conjecture}
		\newtheorem{exmp}[thm]{Example}
		\newtheorem{cor}[thm]{Corollary} 
	
	\theoremstyle{remark} 
		\newtheorem*{rem}{Remark} 
		\newtheorem*{note}{Note} 
		\newtheorem{case}{Case} 
		
\newcommand{\al}{\alpha}
\newcommand{\Q}{\mathbb{Q}}
\newcommand{\R}{\mathbb{Q}}
\newcommand{\C}{\mathbb{C}}
\newcommand{\Z}{\mathbb{Z}}
\newcommand{\F}{\mathbb{F}}
\newcommand{\Ga}{\mathbb{G}}

\newcommand{\aut}[1]{\text{Aut}{(#1)}}

\newcommand{\gener}[1]{\langle #1 \rangle}
\newcommand{\charr}[1]{\text{char}(#1)}
\newcommand{\nth}{n\textsuperscript{th}}

\newcommand{\tworow}[2]{\genfrac{}{}{0pt}{}{#1}{#2}}
\newcommand{\xdeg}[2]{[#1 : #2]}
\newcommand{\gal}[2]{\text{Gal}(#1/#2)}
\newcommand{\minpoly}[2]{m_{#1, #2}(x)}

\newcommand{\mapping}[5]{\begin{align*}
	#1 : \quad     #2 &\rightarrow #3 \\
			#4  &\mapsto #5
\end{align*}	
}

\usepackage{hyperref}

		
\title{MATH 3962 - Rings, Fields and Galois Theory}								% Document Title
\author{Andrew Tulloch}
%\date{}                                           % Activate to display a given date or no date


\begin{document}
\maketitle

\section{Background Theory} % 
\label{cha:background_theory}

\begin{defn}[Monoids]
	A monoid is a set $S$ equipped with a single operation $\cdot$ obeying the following axioms
	\begin{itemize}
		\item \textbf{Closure} For all $a,b \in S$, $a \cdot b \in S$ 
		\item \textbf{Associativity} For all $a,b,c \in S$, $(a \cdot b) \cdot c = a \cdot (b \cdot c)$.
		\item \textbf{Identity}  There exists an element $e$ in $S$ such that $e \cdot a = a = a \cdot e$ for all $a \in S$. 
	\end{itemize}
	
\end{defn}

\begin{defn}[Groups]
	A group is a set equipped with an operation $\cdot$ obeying the axioms of associativity, existence of inverses, and existence of an identity.
\end{defn}

\begin{defn}[Abelian Groups]
	An abelian group is a group where the operation $\cdot$ is commutative.
\end{defn}

\begin{defn}[Cyclic Groups]
	A cyclic group is a group that can be generated by a single element $\langle x \rangle = \{ x^n \, | \, x \in \mathbb{Z}\}$
\end{defn}

\begin{defn}[Subgroup]
	A subset $H$ of a group $G$ is a subgroup of $G$ if and only if $H$ is non-empty and for all $x,y \in G$ 
	\begin{itemize}
		\item If $x,y \in H$ then $x \cdot y \in H$ 
		\item If $x \in H$ then $x^{-1} \in H$.
	\end{itemize}
\end{defn}

\begin{defn}[Normal Subgroup]
	A subgroup $K$ of a group $G$ is said to be \textbf{normal} in $G$ if $g^{-1} k g \in K$ for all $k \in K$ and $g \in G$.  Equivalently, the subgroup $K$ is normal in $G$ if $g^{-1}Kg = K$, or $gK = Kg$ for all $g \in G$.
\end{defn}

\begin{defn}[Quotient Group]
	If $G$ is a group and $H$ is a subgroup, we can form the \textbf{quotient group} $G/H$ as follows.
	Defining the equivalence relation $~$ as follows: for all $x,y \in G$, \[
		x \sim y if x = yh
	\] for some $h \in H$.  The set \[
		xH = \{xh \, | \, h \in H \}
	\] is called the left coset containing $x$.  These cosets partition $G$, and the number of cosets of $H$ in $G$ is denoted by $\xdeg{G}{H}$.  By \textbf{Lagrange's Theorem}, we have \[
		\xdeg{G}{H} = \frac{|G|}{|H|}
	\]
	
	Now, letting $K$ be a normal subgroup of $G$, we have the following.
	THe set of al cosets in $G$ forms a group, with multiplication satisfying $(xK)(yK) = xyK$ for all $x,y \in G$.
\end{defn}


\begin{defn}[Solvable Groups]
	A group $G$ is \textbf{solvable} if there is a chain of subgroups\[
		1 = G_0 \lhd G_1 \lhd G_2 \lhd \cdots \lhd G_s = G
	\] where each $G_i$ is normal in $G_{i+1}$ and the quotient groups $G_{i+1}/G_i$ is abelian for all $i$.
\end{defn}

\begin{cor}
	The finite group $G$ is solvable if and only if for every divisor $n$ of $|G|$ with $\text{gcd}(n, \frac{|G|}{n}) = 1$, $G$ has a subgroup of order $n$.
\end{cor}

\begin{cor} Let $N$ be normal in $G$.
	If $N$ and $G/N$ are solvable, then $G$ is solvable.
\end{cor}


\begin{thm}[Subgroups of Cyclic Groups]
	Let $G = \langle x \rangle$ be a cyclic group.  Then we have the following.
	\begin{itemize}
		\item Every subgroup of $H$ is cyclic.   More precisely, if $K \leq H$, then either $K = \{1\}$ or $K = \langle x^d \rangle$, where $d$ is the smallest positive integer such that $x^d \in K$.
		\item If $|H| = \infty$, then if $a \neq b,$ then $\gener{x^a} \neq \gener{x^b}$. 
		\item If $|H| = n < \infty$, then for each positive integer $a$ dividing $n$ there is a unique subgroup of $H$ of order $a$.  This subgroup is the cyclic group $\gener{x^d}$, where $d = \frac{n}{a}$.  Furthermore, the subgroups of $H$ correspond bijectively with the positive divisors of $n$.
	\end{itemize}
\end{thm}

\begin{defn}[Homomorphism of Groups]
	A map $\varphi: G \rightarrow H$ is a homomorphism if and only if
	\begin{itemize}
		\item $\varphi(xy) = \varphi(x) \varphi(y)$ for all $x,y \in G$
	\end{itemize}
\end{defn}

\begin{defn}[Isomorphism of Groups]
	A map $\varphi: G \rightarrow H$ is an isomorphism of groups if and only if 
	\begin{itemize}
		\item $\varphi$ is a homomorphism.
		\item $\varphi$ is a bijection.
	\end{itemize}
\end{defn}

\begin{defn}[Symmetric Groups]
	The symmetric group of order $n$ is the set of all permutations of the finite set $\{1,2,\dots, n\}$, with the operation being composition of permutations.
\end{defn}

\begin{prop}[Properties of the Symmetric Groups]
	Let $S_n$ be the symmetric group of order $n$.  Then we have 
	\begin{itemize}
		\item  $| S_n | = n!$
		\item $S_n$ is non-abelian for all $n \geq 3$.
	\end{itemize}
	
\end{prop}

\begin{defn}[Elementary Symmetric Functions]
	Let $x_1, x_2, \dots, x_n$ be indeterminates.  Then the \textbf{elementary symmetric functions} $s_1, x_2, \dots, s_n$ are defined by \begin{align*}
		s_1 &= x_1 + x_2 + \cdots  + x_n\\
		s_2 &= x_1 x_2 + x_1 x_3 + \cdots + x_2 x_3 + x_2 x_4 + \cdots x_{n-1} x_n \\
		s_n &= x_1 x_2 \cdots x_n
	\end{align*}
\end{defn}

\begin{defn}[Symmetric functions]
A function $f(x_1, x_2, \dots, x_n)$ is called \textbf{symmetric} if it is not changed by any permutation of the variables $x_1, x_2, \dots, x_n$. 
\end{defn}

\subsection{Solving Polynomial Equations} % 
\label{sub:solving_polynomial_equations}
We have explicit solutions for solving polynomials of degrees two and three. Polynomials of degree two are solved using the quadratic equation.  Polynomials of degree three are solvable using \textbf{Cardano's Method}

% subsection solving_polynomial_equations 

% section background_theory 

\section{General Ring Theory} % 
\label{cha:general_ring_theory}

\begin{defn}[Rings]
	A \textbf{ring} $R$ is a set equipped with two binary operations $+$ and $\times$ satisfying the following axioms
	\begin{itemize}
		\item (R,+) is an \textbf{abelian group} -  impyling the existence of negatives, a zero element, and commutative addition.
		\item $\times$ is associative: $(a \times b) \times c = a \times (b \times c)$ for all $a,b,c \in R$
		\item Multiplication distributes over addition: \[
			(a + b) \times c  = (a \times c) + (b \times c) \qquad c \times (a + b) = (c \times a) + (c \times b)
		\]
	\end{itemize}
	
	A ring is \textbf{commutative} if multiplication is commutative.  
	
	A ring is said to contain an identity if there is an alement $1 \in R$ such that $1 \times a = a \times 1 = a$ for all $a \in R$.
\end{defn}

\begin{defn}[Zero divisors]
	A non-zero element $a \in R$ is called a zero divisor if there is a nonzero element $b \in R$ such that either $ab = 0$ or $ba = 0$. 
\end{defn}

\begin{defn}[Field]
	A field can be defined in several ways.
	\begin{itemize}
		\item A field is a commutative ring $F$ with identity $1 \neq 0$ such that every non-zero element $a \in F$ has a multiplicative inverse.
		\item A field is a commutative ring $F$ with identiy $1 \neq 0$ in which every nonzero element is a unit, i.e. $F^{\times} = F - \{0\}$.
	\end{itemize}
\end{defn}

\begin{defn}[Unit]
	Assume a ring $R$ has identity $1 \neq 0$.  Then an element $u$ of $R$ is called a \textbf{unit} if there is some $v \in R$ such that $uv = vu = 1$.  The set of units in $R$ is denoted $R^{\times}$.  Te set of units in $R$ form a group under multiplication, denoted the \textbf{group of units} of $R$.
\end{defn}

\begin{defn}[Integral Domain]
	An integral domain is a commutative ring with identity $1 \neq 0$ with no zero divisors.
\end{defn}

\begin{cor}[Cancellation property]
	Let $R$ be an integral domain.  Then for any $a,b,c \in R$, if $ab = ac$, then either $a = 0$ or $b = c$. 
		ab = ac 
\end{cor}

\begin{cor}
	Any finite integral domain is a field.
\end{cor}

\begin{defn}[Subring]
	A subring of a ring $R$ is a subgroup of $R$ that is closed under multiplication.  Alternatively, a subset $S$ of a ring $R$ is a subring if the operations of addition and multiplication in $R$ when restricted to $S$ gives $S$ the structure of a ring.    
\end{defn}

\begin{cor}
	To show a subset of a ring $R$ is a subring it suffices to check that it is \textbf{nonempty} and \textbf{closed under subtraction and under multiplication}.
\end{cor}

\begin{defn}[Characteristic of a Ring]
	The characteristic of a ring, $\charr{R}$, is defined as the smallest positive number $n$ such that $ n \times 1 = 0$
\end{defn}

\begin{exmp}[Examples of Rings]
	The set of all $n \times n$ matrices over a ring $R$ is a ring, denoted by $\text{Mat}_n(R)$.
	
	Let $x$ be indeterminate, and let $R$ be a commutative ring with identity $1 \neq 0$.  The set of all formal sums \[
		a_n x^n + a_{n-1} x^{n-1} + \cdots + a_1 x + a_0 
	\] with $a_i \in R$ is the \textbf{polynomial ring} $R[x]$.
\end{exmp}

\subsection{Homomorphisms, kernels, images.} % 
\label{sub:homomorphisms_kernels_images_}
This section deals with maps between rings $R$ and $S$.

\begin{defn}[Homomorphisms of Rings]
	Let $\varphi: R \rightarrow S$ be a map between two rings $R$ and $S$.  Then $\varphi$ is a ring homomorphism if and only if
	\begin{itemize}
		\item $\varphi(a + b) = \varphi(a) + \varphi(b)$
		\item $\varphi(ab) = \varphi(a) \varphi(b)$
		\item $f(1) = 1$
	\end{itemize}
\end{defn}

\begin{cor}
	The image of a ring homomorphism $\varphi$ is a subring of $S$.
\end{cor}
\begin{defn}[Kernel of a Homomorphism]
	The kernel of a ring homomorphism $\varphi$, denoted $\ker \varphi$, is the set of elements $R$ that maps to $0 \in S$, i.e. the set of all $a \in R$ such that $\varphi(a) = 0$.
\end{defn}

\begin{cor}
	The kernel of a homomorphism $\varphi$ is a subring of $R$.  Furthermore, if $\al \in \ker \varphi$, then $r \al$ an $\al r \in \ker \varphi$ for all $r \in R$, i.e, $\ker \varphi$ is closed under multiplication by elements in $R$. 
\end{cor}

\begin{exmp}
	Let $R$ be a subring of a commutative ring $T$, and let $\al \in T$.  Then the function $\text{eval}_\alpha : R[x] \rightarrow T$ is a homomorphism.
\end{exmp}


% subsection homomorphisms_kernels_images_ 

\subsection{Ideals, Quotient Rings, and Isomorphisms} % 
\label{sub:ideals_quotient_rings_and_isomorphisms}


\begin{defn}[Ideal of a Ring]
	A subset $I$ of a ring $R$ is an ideal of $R$ if and only if the following conditions all hold.
	\begin{itemize}
		\item $I$ is nonempty
		\item $a + b \in I$ for al $a, b \in I$
		\item $-x \in I$ for all $x \in I$
		\item $ax, xa \in I$ for all $x \in I$ and $a \in R$.
	\end{itemize}
\end{defn}

\begin{cor}
	The kernel of a homomorphism $\varphi$ is an ideal of $R$.
\end{cor}

\begin{prop}[Cosets of an Ideal]
	Let $I$ e an ideal in a ring $R$.  The equivalence relation $\equiv$ defined by $a \equiv b$ if and only if $a - b \in I$ is an equivalence relation on the ring $R$, partioning the ring into a set of equivalence classes $r + I$ with $r \in R$ called the \textbf{cosets} of $I$ in $R$
\end{prop}

\begin{defn}[Quotient Ring]
	Let $R$ be a ring and let $I$ be an ideal of $R$.  Then the additive quotient group $R / I$ is a ring under the binary operations:
	\begin{itemize}
		\item $(r + I) + (s+I) = (r+s) + I$
		\item $(r + I) \times (s + I) = (rs) + I$
	\end{itemize}
	
	The elements of $R/I$ are precisely the cosets of $I$ in $R$.
\end{defn}

\begin{thm}
	Let $I$ be an ideal in the ring $R$.  Then the mapping $\varphi: R \rightarrow R/I$ given by $\varphi(\alpha) = \al + I$ is a surjective homomorphism with kernel $I$.
\end{thm}

We can collect these results into the following theorem, known as the \textbf{First Isomorphism Theorem}.

\begin{thm}[First Isomorphism Theorem]
	Let $R$ and $S$ be rings, and $\varphi: R \rightarrow S$ a homomorphism of rings.  Then the kernel of $\varphi$ is an ideal of $R$, the image of $\varphi$ is a subring of $S$, and there is an isomorphism $\psi: R / \ker \varphi \rightarrow \varphi(R)$ such that $\psi(r + \ker \varphi) = \varphi(r)$. 
\end{thm}


\begin{thm}[Second Isomorphism Theorem]
	Let $R$ be a ring. Let $S$ be a subring and let $I$ be an ideal of $R$.  Then $S + I = \{ s + i \, | \, s \in S, i \in I \}$ is a subring of $R$, $S \cap I$ is an ideal of $S$, and $(S + I)/I$ is isomorphic to $S/(S \cap I)$.
\end{thm}

\begin{thm}[Third Isomorphism Theorem]
	Let $I$ and $J$ be ideas of $R$ with $I \subseteq J$.  Then $J/I$ is an ideal of $R/I$ and $(R/I)/(J/I)$ is isomorphic to $R/J$. 
\end{thm}

% subsection ideals_quotient_rings_and_isomorphisms 

\subsection{Classification of Ideals} % 
\label{sub:classification_of_ideals}

\begin{prop}[Sum, Product, Intersection of Ideals]
	Let $I$ and $J$ be ideals of $R$.  Then
	\begin{itemize}
		\item The sum of $I$ and $J$, $I + J$, is equal to $\{ a + b \, | \, a \in I, b \in J \}$.
		\item The product of $I$ and $J$, $IJ$, is equal to the set of all finite sums of elements of the form $ab$ with $a \in I$, $b \in J$.
		\item The intersection of ideals, $I \cap J$, is defined simply as $I \cap J$.
	\end{itemize}

	It can be shown that the sum $I + J$ of ideal $I$ and $J$ is the smallest ideal of $r$ containing both $I$ and $J$, and the product $IJ$ is an ideal contained in $I \cap J$, but can be strictly smaller.
\end{prop}

\begin{cor}
	Let $I = a\Z, J = b\Z$.  Then we have $I + J = d\Z$, where $d  = \text{GCD}(a,b)$.  We also have that $IJ = ab \Z$, and $I \cap J = \text{LCM}(a,b)$ 
\end{cor}
\begin{defn}[Principal Ideal]
	Let $R$ be a commutative ring.  An ideal that can be generated by a single element, of the form $aR$ for some $a \in R$, is called a \textbf{principal ideal}. 
\end{defn}

\begin{cor}
	Every ideal in the ring $\Z$ is principal.
\end{cor}

\begin{prop}
	Let $I$ be an ideal of a ring $R$.  Then $I = R$ if and only if $I$ contains a unit
\end{prop}

\begin{prop}
	Let $I$ be an ideal of a commutative ring $R$.  Then $R$ is a field if and only if its only ideals are $0$ and $R$.
\end{prop}

\begin{defn}[Maximal Ideal]
	An ideal $M$ in an arbitary ring $S$ is called a \textbf{maximal ideal} if $M \neq S$ and the only ideals containing $M$ are $M$ and $S$.
\end{defn}

\begin{prop}
	Asume $R$ is commutative.  Then the idela $M$ is a maximal ideal if and only if the quotient ring $R/M$ is a field.
\end{prop}


\begin{defn}[Prime Ideal]
	Assume $R$ is commutative.  An ideal $P$ is called a \textbf{prime ideal} if $P \neq R$ and whenever the product $ab$ of two elements $a,b \in R$ is an element of $P$, then at least one of $a$ and $b$ is an element of $P$.  
	
	The definition is motivated by the following example.  Let $n$ be a nonnegative integer.  Then $n\Z$ is a \textbf{prime} ideal provided $n \neq 1$ and every time the product $ab$ of two integers is an element of $n\Z$, at least one of $a,b$ is an element of $n\Z$. This is equivalent to stating that whenever $n$ divides $ab$, $n$ must divide $a$ or divide $b$.  Thus, $n$ must be prime.  Thus, \textbf{the prime ideals of $\Z$ are simply the ideal $p\Z$ of $\Z$ generated by prime numbers $p$ together with the ideal $0$}.
\end{defn}

\begin{prop}
	Assume $R$ is commutative.  Then the ideal $P$ is a prime ideal in $R$ if and only if the quotient ring $R/P$ is an integral domain.
\end{prop}

\begin{cor}
	Assume $R$ is commutative.  Then every maximal ideal of $R$ is a prime ideal.
\end{cor}
\begin{proof}
	If $M$ is a maximal ideal then $R/M$ is a field.  As a field is an integral domain, we thus have that $R/M$ is an integral domain, and thus $M$ is a prime ideal.
\end{proof}


% subsection classification_of_ideals 
% section general_ring_theory 

\section{Integral Domains} % 
\label{cha:integral_domains}


\begin{defn}[Field of Fractions]
	Let $R$ be a commutative ring.  Let $D$ be any nonempty subset of $R$ that does not contain 0, does not contain any zero divisors, and is closed under multiplication.  Then there is a commutative ring $Q$ with $1$ such that $Q$ contains $R$ as a subring and every element of $D$ is a unit of $Q$.  The ring $Q$ has the following additional properties.
	\begin{itemize}
		\item Every element of $Q$ is of the form $r d^{-1}$ for some $r \in R$ and $d \in D$.   In particular, if $D = R - \{0\}$, then $Q$ is a field.
		\item The ring $Q$ is the \textbf{smallest} ring containing $R$ in which all element of $D$ become units,
 in the following sense - Any ring containing an isomorphic copy of $R$ in which all the elements of $D$ become units must also contain an isomorphic copy of $Q$.	
	\end{itemize}
\end{defn}

\begin{defn}[Divisibility]
	Let $R$ is a commutative ring.  Let $a,b \in R$.  Then we say that $a$ divides $b$ if and only if $b = ca$ for some $c \in R$.  We write $a | b$ if $a$ divides $b$. 
\end{defn}

\begin{defn}[Units, Irreducibles, Primes, Associates]
	Let $R$ be an integral domain - a commutative ring with $1 \neq 0$ with no zero divisors.  Then we have the following.
	\begin{itemize}
		\item An element $a \in R$ is a  \textbf{unit} in $R$ is an element such that there exists $b \in R$ where $ab = ba = 1$.
		\item Suppose $r \in R$ is nonzero and is not a unit.  Then $r$ is called \textbf{irreducible } in $R$ if whenever $r = ab$ with $a,b \in R$, at least one of $a$ or $b$ must be a unit in $R$.  Otherwise, $r$ is said to be \textbf{reducible.}
		\item The nonzero element $p \in R$ is called \textbf{prime} in $R$ if the ideal $(p)$ generated by $p$ is a prime ideal.  Alternatively, if $R$ is a commutative ring, and $p \in R$.  We say that $p$ is \textbf{prime} if it is nonzero and not a unit, and the following condition holds: for all $a,b \in R$, if $p | ab$ then either $p | a$ or $p | b$.
		\item Two elements $a$ and $b$ differing by a unit are said to be \textbf{associate} in $R$ (i.e., $a = ub$ for some unit $u$ in $R$).
	\end{itemize}
\end{defn}

\begin{prop}
	In an integral domain a prime element is always irreducible.
\end{prop}
\begin{proof}
	Suppose $(p)$ is a nonzero prime ideal and $p = ab$.  Then $ab = p \in (p)$, so by the definition of prime ideal one of $a$ or $b$, say $a$, is in $(p)$.  Thus $a = pr$ for some $r$.  This implies $p = a = prb$, and so $rb = 1$.  Thus $b$ is a unit.  This shows that $p$ is irreducible.
\end{proof}

\begin{defn}[Principle Ideal Domains]
	A \textbf{Principle Ideal Domains} (PID) is an integral domain in which every ideal is principal. 
\end{defn}

\begin{defn}[Unique Factorisation Domains]
	A \textbf{Unique Factorisation Domain} (UFD) is an integral domain $R$ in which every nonzero element $r \in R$ which is not a unit has the following two properties:
	\begin{itemize}
		\item $r$ can be written as a finite product of irreducibles $p_i$ of $R$ (not necessarily distinct)
		\item The decomposition above is \textbf{unique up to associates}: if $r = q_1 q_2 \dots q_m$ is another factorisation of $r$ into irreducibles, then $m = n$ and there is a renumbering of the factors so that $p_i$ is associate to $q_i$. 
		\end{itemize}
\end{defn}

\begin{thm}
	Every principle ideal domain is a unique factorisation domain.
\end{thm}

\subsection{Greatest Common Divisor, Euclidean Algorithm} % 
\label{sub:greatest_common_divisor_euclidean_algorithm}
\begin{defn}[Greatest common divisor]
	Let $R$ be a principal ideal domain and $a,b$ nonzero elements of $R$.  An element $d \in R$ is called a \textbf{greatest common divisor} of $a$ and $b$ if 
	\begin{enumerate}
		\item $d | a$ and $d|b$, and
		\item for all $e \in R$, if $e | a$ and $e | b$ then $e | d$.
	\end{enumerate}
\end{defn}

\begin{prop}[Existence and properties of the GCD]
	Let $R$ be a principal ideal domain, and $a,b \in R$ nonzero elements.  Then
	\begin{itemize}
		\item There is an element $d \in R$ which is a greatest common divisor of $a$ and $b$, and every associate of $d$ is also a greatest common divisor of $a$ and $b$.
		\item The greatest common divisor is unique up to associates.
		\item An element $d \in R$ is a greatest common divisor of $a$ and $b$ if and only if $d | a$, $d|b$ and there exist $r,s \in R$ such that $d = ar + bs$.  
		\item An element $d \in R$ is a greatest common divisor of $a$ and $b$ if and only if $aR + bR = dR$. 
	\end{itemize}
\end{prop}


\begin{defn}[Euclidean Algorithm]
	This operation works in Euclidean Domains - domains where we can define a degree function measuring the size of each element.
	Given $a,b \in R$, calculates the GCD of $a$ and $b$.  Operates as follows:
	
	While $b \neq 0$ - set $a,b$ = $b, \text{Rem}(a,b)$
	
	where Rem($a,b$) is the remainder of $a$ upon division by $b$.
\end{defn}

\begin{thm}
	Every Principle Ideal Domain and Unique Factorisation Domain is a Euclidean Domain.
\end{thm}


% subsection greatest_common_divisor_euclidean_algorithm 

\begin{defn}[Gaussian Integers]
	The Gaussian Integers $\Ga$ are defined as $\Z[i]$, the set $\{ a + b i \, | \, a, b \in Z \}$.
\end{defn}

Let $\alpha = a + bi \in \Ga$. Define the norm $N(\alpha) = \al \overline{\al} = a^2 + b^2$.  

We have the following theorem, due to Fermat.

\begin{thm}
	The prime $p$ is the sum of two integer squares, $p = a^2 + b^2, a,b \in \Z$, if and only if $p = 2$ or $p \equiv 1 \bmod 4$.  This representation is essentially unique up to signs and interchanging elements.
	
	Secondly, the irreducible element in in the Gaussian integers $\Ga$ are as follows.
	\begin{itemize}
		\item 1 + i (with norm 2)
		\item The primes $p \in \Z$ with $p \equiv 3 \bmod 4$ (with norm $p^2$)
		\item $a + bi, a-bi$, the distinct irreducible factors of $p = a^2 + b^2$ for primes $p$ with $p \equiv 1 \bmod 4$.
	\end{itemize} 
\end{thm}

\begin{proof}
	COMPLETE THIS!
\end{proof}

\subsection{Polynomial Rings} % 
\label{sub:polynomial_rings}

\begin{defn}[Polynomial Ring]
	Let $x$ be indeterminate, and let $R$ be a commutative ring with identity $1 \neq 0$.  The set of all formal sums \[
		a_n x^n + a_{n-1} x^{n-1} + \cdots + a_1 x + a_0 
	\] with $a_i \in R$ is the \textbf{polynomial ring} $R[x]$.
\end{defn}

\begin{prop}
	Let $R$ be an integral domain.  Then
	\begin{itemize}
		\item $\deg p(x) q(x) = \deg p(x) + \deg q(x)$ if $p(x), q(x)$ are non-zero.
		\item The units of $R[x]$ are the units of $R$.
		\item $R[x]$ is an integral domain.
	\end{itemize}
\end{prop}

\begin{prop}
	Let $I$ be an ideal of the ring $R$, and let $(I) = I[x]$ denote the ideal of $R[x]$ generated by $I$ (the set of all polynomials with coefficients in $I$).  Then we have\[
		R[x]/(I) \simeq (R/I)[x]
	\] 
\end{prop}

\begin{thm}
	Let $F$ be a field.  The polynomial ring $F[x]$ is a Euclidean Domain.  Specifically, if $a(x)$ and $b(x)$  are two polynomials in $F[x]$ with $b(x)$ nonzero, then there are unique $q(x)$ and $r(x)$ in $F[x]$ such that \[
		a(x) = q(x) b(x) + r(x)
	\] with $r(x) = 0$ or $\deg r(x) < \deg b(x)$.
\end{thm}

\begin{thm}
	If $F$ is a field, then $F[x]$ is a Principal Ideal Domain and a Unique Factorisation Domain.
\end{thm}

\begin{thm}[Gauss's Lemma]
	Let $R$ be a UFD with field of fractions $F$ and let $p(x) \in R[x]$.  If $p(x)$ is reducible in $F[x]$ then $p(x)$ is reducible in $R[x]$.
\end{thm}

\begin{cor}
	$R$ is a UFD if and only if $R[x]$ is a UFD.
\end{cor}


\begin{prop}
	Let $F$ be a field and let $p(x) \in F[x]$.  Then $p(x)$ has a factor of degree one if and only if $p(x)$ has a root in $F$, i.e., there exists $\al \in F$ with $p(\al) = 0$.
\end{prop}

\begin{cor}
	A quadratic or cubic in $F[x]$ is reducible if and only if it has a root in $F$.
\end{cor}

Our next theorem gives us conditions on the roots of polynomials with integer coefficients.

\begin{thm}[Rational Roots Theorem]
	Let $p(x) = a_n x^n + a_{n-1} x^{n-1} + \cdots + a_1 x + a_0 \in \Z[x]$.  If $\frac{r}{s} \in \Q$ is a root of $p(x)$ and $r,s$ are relatively prime, then $r | a_0$ and $s | a_n$.  In particular, if $p(x)$ is \textbf{monic} and $p(d) = 0$ for all integers $d$ dividing the constant term $a_0$ of $p(x)$, then $p(x)$ has no roots in $\Q$.
\end{thm}

The following theorem gives conditions on the reducibility of a polynomial modulo some proper ideal.

\begin{thm}
	Let $I$ be a proper ideal in the integral domain $R$ and let $p(x)$ be a non-constant monic polynomial in $R[x]$.  If the image of $p(x)$ in $(R/I)[x]$ cannot be factored in $(R/I)[x]$ into two polynomials of smaller degree, then $p(x)$ is irreducible in $R[x]$.
\end{thm}

\begin{exmp}
	Consider the polynomial $p(x) = x^2 + x + 1 \in \Z[x]$.  Then, reducing modulo 2, we see that $p(x)$ is irreducible in $\Z[x]$.
\end{exmp}

Our next theorem, Eisenstein's Irreducibility Criterion, applied to the ring $\Z[x]$ is stated below.

\begin{thm}[Eisenstein's Criterion for $\Z$]
	
	Let $p$ be a prime in $\Z$ and let $f(x) = a_n x^n + a_{n-1}x^{n-1} + \cdots + a_1 x + a_0 \in \Z[x]$. Suppose $p$ divides $a_i$ for all $a_i, i \in \{0, 1, \dots, n-1 \}$, $p$ does not divide $a_n$, and $p^2$ does not divide $a_0$.  Then $f(x)$ is irreducible in $\Q[x]$.
\end{thm}
% subsection polynomial_rings 


% section integral_domains 


\section{Fields} % 

\label{sec:fields}

\begin{defn}[Field Extension]
	If $K$ is a field containing the subfield $F$, then $K$ is said to be an \textbf{extension field}, or simply and \textbf{extension}, of $F$, denoted $K/F$.  
\end{defn}

\begin{defn}[Degree of an Extension]
	The degree of a field extension $K/F$, denoted $\xdeg{K}{F}$, is the dimension of $K$ as a vector space over $F$.  The extension is said to be finite if $\xdeg{K}{F}$ is finite, and \textbf{infinite} otherwise
\end{defn}

\begin{thm}
	Let $F$ be a field and let $p(x) \in F[x]$ be an irreducible polynomial.  Then there exists a field $K$ containing an isomorphic copy of $F$ in which $p(x)$ has a root.  identifying $F$ with this isomorphic copy shows that there exists an extension of $F$ in which $p(x)$ has a root. 
\end{thm}
\begin{proof}
	 Consider the quotient  \[
		K = F[x]/(p(x))
	\] of the polynomial ring $F[x]$ by the ideal generated by $p(x)$.  As $p(x)$ is irreducible in the PID $F[x]$, the ideal generated by $p(x)$ is a \textbf{maximal} ideal.  Thus, the quotient $F[x]/(p(x))$ is a field.  The projection $\pi$ of $F[x]$ to the quotient $F[x]/(p(x))$ restricted to $F \subset F[x]$ gives a homomorphism $\varphi = \pi |_{F} : F \rightarrow K$ which is not identically zero, and hence $\varphi(F) \simeq F$.  
	
	If $\overline{x} = \pi(x)$ denotes the image of $x$ in the quotient $K$, then we have \begin{align*}
		p(\overline{x}) &= \overline{p(x)}  &\text{(since $\pi$ is a homomorphism)} \\
					&=  p(x) \bmod p(x)	 &\text{in $F[x]/(p(x))$} \\
					&= 0
	\end{align*}
	
	Thus $K$ contains a root of the polynomial $p(x)$.  Hence, $K$ is an extension of $F$ in which the polynomial $p(x)$ has a root. 
\end{proof}

Our next theorem allows us to understand the field $K = F[x]/(p(x))$ more fully, by having a simple representation for the elements of this field.  Since $F$ is a subfield of $K$, we might ask in particular for a basis for $K$ as a vector space over $F$.

\begin{thm}
	Let $p(x) \in F[x]$ be an irreducible polynomial of degree $n$ over the field $F$, and let $K$ be the field $F[x]/(p(x))$.  Let $\theta = x \bmod (p(x)) \in K$.  Then the elements \[
		1, \theta, \theta^2, \dots, \theta^{n-1}
	\]
	are a basis for $K$ as a vector space over $F,$ so the degree of the extension is $n$, i.e., $\xdeg{K}{F} = n$. Hence, \[
		K = \{ a_0 + a_1 \theta + a_2 \theta^2  + \cdots + a_{n-1} \theta^{n-1} \,| \, a_i \in F \}
	\] consists of all polynomials of degree less than or equal to $n$ in $\theta$.
\end{thm}

\begin{proof}
	Let $a(x) \in F[x]$ be any polynomial with coefficients in $F$.  Since $F[x]$ is a Euclidean Domain, we may divide $a(x)$ by $p(x)$:\[
		a(x) = q(x)p(x) + r(x)
	\] 
	It thus follows that $a(x) \equiv r(x) \bmod (p(x))$, which shows that every residue class in $F[x]/(p(x))$ is represented by a polynomial of degree less than $n$. Hence the images $1, \theta, \theta^2,\dots, \theta^{n-1}$ of $1,x,x^2,\dots$ in the quotient \textbf{span} the quotient as a vector space over $F$.   We now show these elements are linearly independent, and so form a basis for the quotient over $F$.
	If the elements $1, \theta, \theta^2, \dots, \theta^{n-1}$ were not linearly independent in $K$, then there would be a linear combination\[
		b_0 + b_1 \theta + b_2 \theta^2 + \cdots + b_{n-1} \theta^{n-1} = 0
	\] in $K$, with $b_i \in F$ not all equal to zero.  This is equivalent to \[
	b_0 + b_1 \theta + b_2 \theta^2 + \cdots + b_{n-1} \theta^{n-1} \equiv 0 \bmod (p(x))
	\] i.e., $p(x)$ divides the above polynomial in $x$.  But $\deg p(x) > \deg{\sum^{n-1} b_i x^i}$, and so by contradiction we have the above elements are a basis for $K$ over $F$.  Thus $\deg{K}{F} = n$.
\end{proof}

\begin{prop}
	The above theorem gives us a formula for elements of the field $K$.  Let $K$ be an extension of $F$, and $a(\theta), b(\theta) \in K$.  Then addition is defined as usual, and multiplication in $K$ is defined as \[
		a(\theta) b(\theta) = r(\theta)
	\]  where $r(x)$ is the remainder obtained upon dividing the polynomial $a(x) b(x)$ by $p(x)$ in $F[x]$
\end{prop}

\begin{defn}[Simple Extension]
	If the field $K$ is generated by a single element $\al$ over $F,$ then $K = F(\al)$, then $K$ is said to be a \textbf{simple} extension of $F$ and the element $\al$ is called a \textbf{primitive element} for the extension.
\end{defn}

\begin{thm}
	Let $F$ be a field and $p(x) \in F[x]$ be an irreducible polynomial. Suppose $K$ is an extension field of $F$ containing a root $\al$ of $p(x)$, thus $p(\al) = 0$.  Let $F(\al)$ denote the subfield of $K$ generated over $F$ by $\al$.  Then \[
		F(\al) \simeq F[x]/(p(x))
	\]
\end{thm}
\begin{proof}
	Consider the natural homomorphism
	\mapping{\varphi}{F[x]}{F(\alpha) \subseteq K}{a(x)}{a(\alpha)}
	Since $p(\alpha) = 0$ by assumption, we have that the element $p(x)$ is in the kernel of $\varphi$, and so we obtain an induced homomorphism \[
		\varphi : F[x]/(p(x)) \rightarrow F(\al)
	\]
	Since $p(x)$ is irreducible, we have that the quotient ring is a field, and as $\varphi$ is not identically zero, we must have $\varphi$ is an isomorphism.
\end{proof}


We now prove a theorem regarding the different roots of an irreducible polynomial.  Consider the equation $p(x) = x^3 - 2$.  Adjoining any of the three roots produces the same field extension (up to isomorphism).    This is known as the \textbf{Isomorphism Extension Theorem}.   

\begin{thm}[Isomorphism Extension Theorem]
	Let $\varphi : F \mapsto F'$ be an isomorphism of fields.  Let $p(x) \in F[x]$ be irreducible and let $p'(x) \in F'[x]$ be the irreducible polynomial obtained by applying the map $\varphi$ to the coefficients of $p(x)$.  Let $\alpha$ be a root of $p(x)$ (in some extension of $F$), and let $\beta$ be a root of $p'(x)$ in some extension of $F'$.  Then, there is an isomorphism
	\mapping{\sigma}{F(\alpha)}{F'(\beta)}{\alpha}{\beta}
	mapping $\alpha$ to $\beta$ and extending $\varphi$, i.e., such that $\sigma$ restricted to $F$ is the isomorphism $\varphi$.
\end{thm}
\begin{proof}
	The isomorphism $\varphi$ induces a natural isomorphism from $F[x]$ to $F'[x]$ which maps the maximal ideal $(p(x))$ to the maximal ideal $(p'(x))$.  Taking quotients by these ideals, we have the following isomorphism of fields\[
		F[x]/(p(x)) \rightarrow F'[x]/(p'(x))
	\]
	and as the above fields are isomorphic to $F(\alpha)$ and $F'(\beta)$, respectively.
\end{proof}

In the following, let $K$ be an extension of $F$.
\begin{defn}[Algebraic Elements and Algebraic Extensions]
	An element $\alpha$ in $K$ is said to be \textbf{algebraic} over $K$ if $\alpha$ is a root of some nonzero polynomial $f(x) \in F[x]$.  If $\alpha$ is not algebraic then it is \textbf{transcendental} over $F$.  The extension $K/F$ is said to be \textbf{algebraic} if every element of $K$ is algebraic over $F$.
\end{defn}

\begin{prop}[Minimal polynomials]
	Let $\alpha$ be algebraic over $F$.  Then there is a unique monic irreducible polynomial $\minpoly{\alpha}{F} \in F[x]$ which has $\alpha$ as a root. A polynomial $f(x) \in F[x]$ has $\al$ as a root if and only if $\minpoly{\al}{F}$ divides $f(x)$ in $F[x]$.
\end{prop}
\begin{proof}
	Let $g(x) \in F[x]$ be a polynomial of minimal degree having $\alpha$ as a root.  Multiplying $g(x)$ by a constant, we have $g(x)$ is monic.  Supposing the $g(x)$ were reducible in $F[x]$, then \[
		g(x) = a(x)b(x)
	\] with $a(x), b(x)$ having degrees less than $\deg g(x)$.  Yet as $g(\alpha) = 0$, then either $a(\al)$ or $b(\al)$ are zero, contradicting the minimal degree of $g(x)$.
	
	Suppose now that $f(x) \in F[x]$ is a polynomial having $\al$ as a root.  By the Euclidean Algorithm in $F[x]$, there are polynomials $q(x), r(x) \in F[x]$ such that \[
		f(x) = q(x) g(x) + r(x)
	\] with $\deg r(x) < \deg g(x)$.  Then $f(\al) = g(\al) q(\al) + r(\al) = r(\al) = 0$, and thus $r(x) = 0$ by minimality of $g(x)$.  Thus, any polynomial $f(x) \in F[x]$ with root $\al$ is divisible by $g(x)$.  
	This proves that $\minpoly{\alpha}{F} = g(x)$, completing the proof.
\end{proof}

\begin{cor}
	If $L/F$ is an extension of fields and $\al$ is algebraic over both $F$ and $L$, then $\minpoly{\al}{L}$ divides $\minpoly{\al}{F}$ in $L[x]$.
\end{cor}

\begin{prop}
	Let $\al$ be algebraic over $F$, and let $F(\al)$ be the field generated by $\al$ over $F$. Then\[
		F(\al) \simeq F[x]/(\minpoly{\al}{F})
	\]
	so that in particular,\[
		\xdeg{F(\al)}{F} = \deg \minpoly{\al}{F} = \deg \al
	\]
\end{prop}

\begin{thm}[Tower Theorem]
	Let $F \subseteq K \subseteq L$ be fields. Then\[
		\xdeg{L}{F} = \xdeg{L}{K} \xdeg{K}{F}
	\]
\end{thm}
\begin{proof}
	The proof proceeds as follows.  Let $(\alpha_i)$ be a basis for $L$ over $K$, and let $(\beta_i)$ be a basis for $K$ over $F$. The elements of $L$ are of the form $\sum a_i \alpha_i$, with $a_i \in K$.  Similarly, the elements $a_i$ are of the form $\sum b_i \beta_i$ with $b_i \in F$.  Thus, elements of $L$ are of the form $\sum c_{ij} \alpha_i \beta_j$ - thus $\alpha_i \beta_j$ span $L$.  Now, consider the linear relation $\sum c_{ij} \al_i \beta_j = 0$.  As the elements $\beta_j$ and $\alpha_i$ are both basis, it can be shown that $c_{ij} = 0$, thus elements $\alpha_i \beta_j$ are a basis for $L$ over $F$, and the theorem follows.
\end{proof}

\begin{defn}[Splitting Field]
	Let $F$ be a field.  The extension field $K$ of $F$ is called splitting field for the polynomial $f(x) \in F[x]$ if $f(x)$ factors completely into linear factors (or \textbf{splits completely}) in $K[x]$ and $f(x)$ does not factor completely into linear factors over any proper subfield of $K$ containing $F$.
\end{defn}

\begin{thm}
	For any field $F$, if $f(x) \in F[x]$, then there exists and extension $K$ of $F$ which is a splitting field for $f(x)$.
\end{thm}

\begin{prop}
	The splitting field for a polynomial of degree $n$ over $F$ is of degree at most $n!$ over $F$.
\end{prop}


\begin{prop}[Uniqueness of splitting fields]
	Any two splitting fields for a polynomial $f(x) \in F[x]$ over a field $F$ are isomorphic.
\end{prop}

\begin{proof}
	Take $\varphi$ to be the identity mapping from $F$ to itself and $E, E'$ in the isomorphism extension theorem to be the two spitting fields for $f(x)$. 
	
	The proof proceeds by inducting on $n$, the degree of the extension of the splitting field.  
\end{proof}


\begin{defn}[Separable Polynomial]
	A polynomial over $F$ is called \textbf{separable} if it has no multiple roots.  A polynomial which is not separable is inseparable.
\end{defn}

\begin{cor}
	Every irreducible polynomial over a field of characteristic 0 (e.g. $\Q, \Z, \R$) is separable.  A polynomial over such a field is separable if and only if it is the product of distinct irreducible polynomials.
\end{cor}


\subsection{Finite Fields} % 
\label{sub:finite_fields}

A finite field  $\F$ is a field with a finite number of elements.  A finite field has characteristic $p$ for some prime $p$, and so is a finite dimensional vector space over $\F_p$. If the dimension of the extension $\xdeg{\F_p}{\F} = n$, then the finite field has $p^n$ elements.
\begin{prop}
	Let $F$ be a field of characteristic $p$. Then for any $a,b \in F$,\[
		(a+b)^p = a^p + b^p \quad \text{and} \quad (ab)^p = a^p b^p
	\]
	Alternatively, the map $\varphi(a) = a^p$ is an injective field homomorphism from $F$ to $F$
\end{prop}
\begin{proof}
	Use the binomial theorem, and note that ${p \choose i}, i = 1,2,\dots,p-1$ is zero in characteristic $p$.
\end{proof}
\begin{defn}
	The map $\varphi(a) = a^p$ is called the \textbf{Frobenius endomorphism} of $F$.
\end{defn}

\begin{cor}
	If $\F$ is finite of characteristic $p$, then every element of $\F$ is a $p$\textsuperscript{th} power in $\F$ - notationally, $\F = \F^p$
\end{cor}
\begin{proof}
	This follows from the injectivity of the Frobenius endomorphism - as $\F$ is finite, an injective function is surjective.
\end{proof}

The field $K$ is said to be \textbf{separable} over $F$ if eery element of $K$ is the root of a separable polynomial over $F$.  Equivalently, the minimal polynomial over $F$ of every element of $K$ is separable.



% subsection finite_fields 


\subsection{Cyclotomic Extensions} % 
\label{sub:cyclotomic_extensions}
\begin{defn}[Cyclotomic Extensions]
	The extension $\Q(\zeta_n)/Q$ generated by the $n$\textsuperscript{th} roots of unit over $\Q$ is called a cyclotomic extension.
\end{defn}

\begin{defn}[Cyclotomic Polynomial]
	The $\nth$ cyclotomic polynomial $\varphi_n(x)$ is defined as the polynomial whose roots are the primitive $\nth$ roots of unity, which is of degree $\varphi(n)$.  
\end{defn}
\begin{exmp}
	For $p$ prime, the $p$\textsuperscript{th} cyclotomic polynomial $\Phi_p(x)$ is given by \[
		\Phi_p(x) = \frac{x^p - 1}{x-1}
	\] 
	For example, $\Phi_5(x) = x^4 + x^3 + x^2 + x + 1$.
\end{exmp}
\begin{cor}
	We have that \[
		\Phi_n(x) = \frac{x^n - 1}{\Pi_{d|n} \Phi_d(x)}
	\]
\end{cor}

\begin{thm}
	The cyclotomic polynomial $\Phi_n(x)$ is an irreducible monic polynomial in $\Z[x]$ of degree $\varphi(n)$.
\end{thm}
\begin{proof}
	If $\Phi_n(x)$ is reducible, there exist $f(x), g(x) \in \Z[x]$ such that \[
		\Phi_n(x) = f(x) g(x)
	\] where we take $f(x)$ to be an irreducible factor of $\Phi_n(x)$.  Let $\zeta$ be a primitive $\nth$ root of 1 which is a root of $f(x)$ (so that $f(x)$ is the minimal polynomial of $\zeta$ over $\Q$) and let $p$ denote \textbf{any} prime not dividing $n$.  Then $\zeta^p$ is also a primitive $\nth$ root of 1, and hence is a root of either $f(x)$ or $g(x)$.
	
	Suppose $g(\zeta^p) = 0$.  Then $\zeta$ is a root of $g(x^p)$, and since $f(x)$ is the minimal polynomial for $\zeta$, $f(x)$ must divide $g(x^p)$ in $\Z[x]$, say \[
		g(x^p) = f(x) h(x)
	\] Reducing modulo $p$ gives \[
		\overline{g}(x^p) = \overline{f}(x^p) \overline{h}(x^p)
	\] in $\F_p[x]$. 
	
	By the remarks of polynomials over finite fields, we have\[
		\overline{g}(x^p) = (\overline{g}(x))^p
	\] so we have the equation \[
		(\overline{g}(x))^p = \overline{f}(x)\overline{h}(x)
	\] in the UFD $\F_p[x]$.  it follows that $\overline{f}(x)$ and $\overline{g}(x)$ have a factor in common in $\F_p[x]$.
	
	Now, from $\Phi_n(x) = f(x)g(x)$ we see by reducing modulo $p$ that $\overline{\Phi_n}(x) = \overline{f}(x) \overline{g}(x)$, and so we have that $\overline{\Phi_n}(x)$ has a multiple root.  But then also $x^n - 1$ would have a multiple root over $\F_p$ since it has $\overline{\Phi_n}(x)$ as a factor.  This is a contradiction since we have that $x^n - 1$ has $n$ distinct roots over any field of characteristic not dividing $n$. 
	
	Hence $\zeta^p$ must be a root of $f(x)$.  Since this applies to every root $\zeta$ of $f(x)$, we must have that $\zeta^a$ is a root of $f(x)$ for every integer $a$ relatively prime to $n$.  This means that \textbf{every} primitive $\nth$ root of unity is a root of $f(x)$, and thus $f(x) = \Phi_n(x)$, showing the $\Phi_n(x)$ is irreducible.
\end{proof}

% subsection cyclotomic_extensions 


\subsection{Constructible Numbers} % 
\label{sub:constructible_number}

We now explore the set of numbers that can be constructed by a ruler and compass.  Constructible numbers are completely classified by the following theorem.

\begin{prop}
	If the element $\alpha \in \R$ is obtained from a field $F \subset \R$ by a series of compass and straightedge constructions then $\xdeg{F(\alpha)}{F} = 2^k$ for some integer $k \geq 0$.
\end{prop}

\begin{proof}
	If a number is constructible, then there is a chain of subfields $\Q \subset F_0 \subseteq F_1 \subseteq \cdots \subseteq F_m$ such that each field $F_i$ is constructed by adjoining the square root of an element $a_{i-1} \in F_{i-1}$, i.e. $F_{i} = F_{i-1}(\sqrt{a_{i-1}})$.  This is an extension of degree one (if $a_{i-1}$ is a square in $F_{i-1}$) or of degree two. Thus, by the Tower Theorem, we have that the overall extension degree is a power of two, and the proposition follows.
\end{proof}
% subsection constructible_number 


% section fields 


\section{Galois Theory} % 
\label{cha:galois_theory}

\begin{defn}[Automorphism]
	Let $K$ be a field.  
	\begin{itemize}
		\item An isomorphism $\sigma$ of $K$ with itself is called an \textbf{automorphism} of $K$.  The collection of automorphisms of $K$ is denoted $\aut{K}$.  
		\item An automorphism $\sigma \in \aut{K}$ is said to \textbf{fix} an element $\alpha \in K$ if $\sigma(\alpha) = \alpha$.  If $F$ is a subset of $K$, then an automorphism $\sigma$ is said to \textbf{fix} $F$ if it fixes all the elements of $F$. 
	\end{itemize}
\end{defn}

\begin{defn}[]
	Let $K/F$ be an extension of fields.  Let $\aut{K/F}$ be the collection of automorphisms of $K$ which fix $F$.
\end{defn}

\begin{prop}
	$\aut{K}$ is a group under composition and $\aut{K/F}$ is a subgroup.
\end{prop}

\begin{prop}
	Let $K/F$ be a field extension and let $\al \in K$ be algebraic over $F$.  Then for any $\sigma \in \aut{K/F}$,  $\sigma (\al)$ is a root of the minimal polynomial for $\al$ over $F$, i.e., $\aut{K/F}$ permutes the roots of irreducible polynomials.  Equivalently, any polynomial with coefficients in $F$ with $\al$ as a roots has $\sigma(\alpha)$ as a root.
\end{prop}
\begin{proof}
	The proof is simple by noting that $\sigma$ is a homomorphism fixing $F$.
\end{proof}

\begin{prop}
	Let $H \leq \text{Aut}(K)$ be a subgroup of the group of automorphisms of $K$.  Then the collection $F$ of elements of $K$ fixed by all the elements of $H$ is a subfield of $K$.
\end{prop}

\begin{prop}
	Let $E$ be splitting field over $F$ of the polynomial $f(x) \in F[x]$.  Then \[
		|\aut{E/F}| \leq \xdeg{E}{F}
	\] with equality if $f(x)$ is separable over $F$.
\end{prop}

\begin{defn}[Galois Extension]
	Let $K/F$ be a finite extension.  Then $K$ is said to be \textbf{Galois} over $F$ and $K/F$ is a \textbf{Galois} extension if $|\aut{K/F}| = \xdeg{K}{F}$.  If $K/F$ is Galois, then the group of automorphism $\aut{K/F}$ is called the \textbf{Galois group of $K/F$}, denoted $\gal{K}{F}$
\end{defn}

\begin{prop}
	If $K$ is the splitting field over $F$ of a separable polynomial $f(x)$ then $K/F$ is Galois.
\end{prop}

\begin{cor}
	The splitting field of any polynomial over $\Q$ is Galois, since the splitting field of $f(x)$ is clearly the same as the splitting field of the product of the irreducible factors of $f(x)$.
\end{cor}

\begin{defn}[Galois group of a polynomial]
	If $f(x)$ is a separable polynomial over $F$, then the \textbf{Galois group of $f(x)$ over $F$} is the Galois group of the splitting field of $f(x)$ over $F$.
\end{defn}

We now prove a fundamental relation between the orders of subgroups of the automorphism group of a field$K$ and the degrees of the extensions over their fixed fields.
\begin{thm}
	Let $G$ be a subgroup of the automorphisms of a field $K$ and let $F$ be the fixed field.  Then \[
		\xdeg{K}{F} = |G|
	\]
	
\end{thm}
\begin{prop}
	Let $K/F$ be any finite extension.  Then \[
		| \aut{K/F} \leq \xdeg{K}{F}
	\] with equality if and only if $F$ is the fixed field of $\aut{K/F}$.  Alternatively, $K/F$ is Galois if and only if $F$ is the fixed field of $\aut{K/F}$.
\end{prop}

\begin{thm}
	The extension $K/F$ is Galois if and only if $K$ is the splitting field of some separable polynomial over $F$. Furthermore, if this is the case then every irreducible polynomial with coefficients in $F$ which has a root in $K$ is separable and has all its roots in $K$.
\end{thm}

\begin{thm}[Fundamental Theorem of Galois Theory] Let $K/F$ be a Galois extension and set $G = \gal{K}{F}$.  Then there is a bijection from (subfields $E$ of $K$ containing $F$) and (subgroups $H$ of $G$), given by the correspondences (the field $E$ mapping to the elements of $G$ fixing $E$) and ($H$ mapping to the fixed field of $H$) which are inverse to each other.
	
	Let $F \subseteq E \subseteq K$ and $1 \leq H \leq G = \gal{K}{F}$, where $E$ is the fixed field of $H$. Under this correspondence,
	\begin{itemize}
		\item $\xdeg{K}{E} = |H|$ and $\xdeg{E}{F} = |G:H|$, the index of $H$ in $G$.
		\item $K/E$ is always Galois, with Galois group $\gal{K}{E} = H$.
		\item $E$ is Galois over $F$ if and only if $H$ is a normal subgroup in $G$.  If this is the case, then the Galois group is isomorphic to the quotient group \[
			\gal{E}{F} \simeq G/H
		\]
	\end{itemize}
\end{thm}


\begin{defn}[Cyclic Extensions]
	An extension $K/F$ is said to be \textbf{cyclic} if it is Galois with a cyclic Galois group.
\end{defn}

\begin{prop}
	Let $F$ be a field of characteristic not dividing $n$ which contains the $n$\textsuperscript{th} roots of unity.  Then the extension $F(\sqrt[n]{a})$ for $a \in F$ is cyclic over $F$ of degree dividing $n$.
\end{prop}
\begin{proof}
	The extension is Galois over $F$ if $F$ contains the $n$\textsuperscript{th} roots of unity since it is the splitting field for $x^n - a$.  For any $\sigma \in \gal{K}{F}$, $\sigma(\sqrt[n]{a})$ is another root of this polynomial, hence $\sigma(\sqrt[n](a)) = \zeta_\sigma \sqrt[n]{a}$ for some root of unity $\zeta_\sigma$.  This gives a map \mapping{\varphi}{\gal{K}{F}}{\mu_n}{\sigma}{\zeta_\sigma}
	where $\mu_n$ denotes the group of $n$\textsuperscript{th} roots of unity.  Since $F$ contains $\mu_n$, every $n$\textsuperscript{th} root of unity is fixed by every element of $\gal{K}{F}$.  Hence \begin{align*}
		\sigma \tau (\sqrt[n]{a}) &= \sigma(\zeta_\tau \sqrt[n]{a}) \\
									&= \zeta_\tau \sigma(\sqrt[n]{a}) \\
									&= \zeta_\tau \zeta_\sigma \sqrt[n]{a} \\
									&= \zeta_\sigma \zeta_\tau \sqrt[n]{a}
	\end{align*}
	which shows that $\zeta_{\sigma \tau} = \zeta_\sigma \zeta_\tau$, so the map above is a homomorphism.  The kernel consists of precisely of the automorphism which fix $\sqrt[n]{a}$, namely the identity.  This gives an injection of $\gal{K}{F}$ into the cyclic group $\mu_n$ of order $n$, which proves the proposition.
\end{proof}

\begin{thm}
	Any cyclic extension of degree $n$ over a field $F$ of characteristic not dividing $n$ which contains the $n$\textsuperscript{th} roots of unity is of the form $F(\sqrt[n]{a})$ for some $a \in F$.
\end{thm}

\begin{defn}[Solvable by radicals]
	An element $\al$ which is algebraic over $F$ can be \textbf{expressed by radicals} or \textbf{solved for in terms of radicals} if $\alpha$ is an element of a field $K$ which can be obtained by a succession of simple radical extensions
\[
	F = K_0 \subset K_1 \subset \cdots \subset K
\]
where $K_{i+1} = K_i(\sqrt[n_i]{a_i})$ for some $a_i \in K_i$, and $\sqrt[n_i]{a_i}$ is some root of the polynomial $x^{n_i} - a_i$.  Such a field $K$ is a \textbf{root extension} of $F$. 

A polynomial $f(x) \in F[x]$ can be \textbf{solved by radicals} if all its roots can be solved for in terms of radicals.
\end{defn}

\begin{defn}[Solvable Groups]
	A group $G$ is \textbf{solvable} if there is a chain of subgroups\[
		1 = G_0 \lhd G_1 \lhd G_2 \lhd \cdots \lhd G_s = G
	\] where each $G_i$ is normal in $G_{i+1}$ and the quotient groups $G_{i+1}/G_i$ is abelian for all $i$.
\end{defn}

\begin{cor}
	The finite group $G$ is solvable if and only if for every divisor $n$ of $|G|$ with $\text{gcd}(n, \frac{|G|}{n}) = 1$, $G$ has a subgroup of order $n$.
\end{cor}

\begin{cor} Let $N$ be normal in $G$.
	If $N$ and $G/N$ are solvable, then $G$ is solvable.
\end{cor}

\begin{thm}[Solvability of a polynomial by radicals]
	A polynomial $f(x)$ is solvable by radicals if and only if its Galois group is a solvable group.
\end{thm}

\begin{proof}
	IMPORTANT PROOF
\end{proof}

\begin{cor}
	The general equation of degree $n$ cannot be solved by radicals for $n \geq 5$.  For $n \geq 5$ the group $S_n$ is not solvable.
\end{cor}
% section galois_theory 









\end{document}