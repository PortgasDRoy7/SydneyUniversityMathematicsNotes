% Created by Andrew Tulloch

%!TEX TS-program = xelatex
%!TEX encoding = UTF-8 Unicode

\documentclass[10pt, oneside, reqno]{amsart}
\usepackage{geometry, setspace, graphicx, enumerate}
\onehalfspacing                 
\usepackage{fontspec,xltxtra,xunicode}
\defaultfontfeatures{Mapping=tex-text}

% AMS Theorems
\theoremstyle{plain}% default 
\newtheorem{thm}{Theorem}[section] 
\newtheorem{lem}[thm]{Lemma} 
\newtheorem{prop}[thm]{Proposition} 
\newtheorem*{cor}{Corollary} 

\theoremstyle{definition} 
\newtheorem{defn}[thm]{Definition}
\newtheorem{conj}[thm]{Conjecture}
\newtheorem{exmp}[thm]{Example}

\theoremstyle{remark} 
\newtheorem*{rem}{Remark} 
\newtheorem*{note}{Note} 
\newtheorem{case}{Case} 




\newcommand{\expc}[1]{\mathbb{E}\left[#1\right]}
\newcommand{\var}[1]{\text{Var}\left(#1\right)}
\newcommand{\cov}[1]{\text{Cov}\left(#1\right)}
\newcommand{\prob}[1]{\mathbb{P}(#1)}
\newcommand{\given}{ \, | \,}
\newcommand{\us}{0 \leq u \leq s}
\newcommand{\ts}[1]{\{ #1 \}}

\newcommand{\al}{\alpha}
\newcommand{\Q}{\mathbb{Q}}
\newcommand{\R}{\mathbb{R}}
\newcommand{\Com}{\mathbb{C}}
\newcommand{\Z}{\mathbb{Z}}
\newcommand{\F}{\mathbb{F}}
\newcommand{\Ga}{\mathbb{G}}

\newcommand{\dmu}{\, d \mu}

\newcommand{\aut}[1]{\text{Aut}{(#1)}}

\newcommand{\gener}[1]{\langle #1 \rangle}
\newcommand{\charr}[1]{\text{char}(#1)}
\newcommand{\nth}{n\textsuperscript{th}}

\newcommand{\tworow}[2]{\genfrac{}{}{0pt}{}{#1}{#2}}
\newcommand{\xdeg}[2]{[#1 : #2]}
\newcommand{\gal}[2]{\text{Gal}(#1/#2)}
\newcommand{\minpoly}[2]{m_{#1, #2}(x)}

\renewcommand{\phi}{\varphi}

\newcommand{\mapping}[5]{\begin{align*}
    #1 : \quad     #2 &\rightarrow #3 \\
            #4  &\mapsto #5
\end{align*}    
}
        
\usepackage{hyperref}
        
\title{MATH 3969 - Measure Theory and Fourier Analysis}                             % Document Title
\author{Andrew Tulloch}
%\date{}                                           % Activate to display a given date or no date


\begin{document}
\maketitle \tableofcontents \clearpage

 
\section{Measure Theory}
\newcommand{\sig}{$\sigma$-algebra }
\newcommand{\siga}{\mathcal{A}}

\begin{defn}[$\sigma$-algebra]
    Let $X$ be a set.  A collection $\mathcal{A}$ of subsets of $X$ is called a \sig if 
    \begin{itemize}
        \item $\emptyset \in \siga$ 
        \item If $A \subseteq X$ is in $\siga$, then its complement $A^c = X \backslash A$ is in $\siga$
        \item Whenever $A_0,A_1,\dots$ are subsets of $X$ in $\siga$, then their union\[
            \bigcup_{k=0}^\infty A_k
        \] also belongs to $\siga$.
    \end{itemize}
\end{defn}

\begin{defn}[Measure]
    Let $X$ be a set, and let $\siga$ be a \sig of subsets of $X$.   Suppose that $\mu: \siga \rightarrow [0,\infty]$ is a function.  Then $\mu$ is a measure if 
    \begin{itemize}
        \item $\mu(\emptyset) = 0$
        \item Whenever $A_0, A_1,\dots$ are \emph{pairwise disjoint} subsets of $X$ in $\siga$, then \[
            \mu(\bigcup_{k=0}^\infty A_k) = \sum_{k=0}^\infty \mu(A_k)
        \]
    \end{itemize}
\end{defn}

\begin{prop}[Properties of a \sig] Let $\siga$ be a \sig of subsets of a set $X$.  Then
    \begin{itemize}
        \item $X, \emptyset \in \siga$
        \item If $A_k \in \siga$, then $\bigcap_{k=0}^\infty A_k \in \siga$
        \item If $A,B \in \siga$, then $A \cup B, A \cap B \in \siga$
    \end{itemize}
    
\end{prop}

\begin{defn}[Algebra]
    A collection $\siga$ of subsets $A$ of $X$ which satisfies the first two conditions of a \sig and also 
    \begin{itemize}
        \item If $A, B \in \siga$, then $A \cup B \in \siga$
    \end{itemize}
    is called an algebra.  Every \sig is an algebra, but not every algebra is a \sig
\end{defn}

\begin{defn}[\sig generated by $\mathcal{S}$]
    Let $\mathcal{S}$ be a collection of subsets of $X$.  Let \[
        \siga(\mathcal{S}) = \bigcap \{ \siga : \siga \text{ is a \sig}, \text{ and } \mathcal{S} \subseteq \siga \}
    \]
$   \siga(\mathcal{S})$ is called the \emph{\sig generated by $\mathcal{S}$}
\end{defn}

\begin{defn}[Borel \sig]
    Let $X$ be a metric space and $\mathcal{S}$ the collection of all open sets in $X$.  We call $\mathcal{B} = \siga(\mathcal{S})$ the \emph{Borel \sig}.  Sets in $\mathcal{B}$ are called \emph{Borel sets}.
\end{defn}
\begin{cor}
    We have the following examples of Borel sets.
    \begin{itemize}
        \item Any open set is a Borel set.
        \item If $B$ is a Borel set, then so is $B^c$.
        If $B_0, B_1,\dots$ is a sequence of Borel sets, then so are $\bigcup_{k=0}^\infty B_k$ and $\bigcap_{k=0}^\infty B_k$.
    \end{itemize}
\end{cor}
\subsection{Properties of Measures} % (fold)
\label{sub:properties_of_measures}


\begin{prop}[The Monotonicity Property]
    If $A$ and $B$ are $\mu$-measurable subsets of $X$ with $A \subseteq B$, then $\mu(A)\leq \mu(B)$.
\end{prop}

\begin{prop}[The Countable Subadditivity Property]
    If $A_0,A_1,\dots$ are $\mu$-measurable subsets of $X$, then \[
        \mu(\bigcup_{k=0}^\infty A_k) \leq \sum_{k=0}^\infty \mu(A_k)
    \]
\end{prop}

\begin{prop}[Monotone Convergence Property of Measures]
    Let $A_0 \subseteq A_1 \subseteq A_2$ be an increasing sequence of measurable sets.  Then \[
        \mu(\bigcup_{k=0}^\infty A_k) = \lim_{k \rightarrow \infty} \mu(A_k)
    \]
\end{prop}

\begin{prop}
    Let $A_0 \supseteq A_1 \supseteq A_2 \dots$ be sets from some \sig $\mathcal{A}$.  If $\mu(A_0) < \infty$, then 
    \[
    \mu(\bigcap_{k = 0}^\infty A_k) = \lim_{k \rightarrow \infty} \mu_(A_k)
    \]
\end{prop}

% subsection properties_of_measures (end)

\subsection{Constructing $\sigma$-algebras and measures} % (fold)
\label{sub:constructing_sigma_algebras_and_measures}
\begin{defn}[Lebesgue outer measure]
    If $A \subseteq \R$, let \[
        m^*(A) = \inf \left\{ \sum_{k=0}^\infty (b_k - a_k) \given a_k < b_k, A \subseteq \bigcup_{k=0}^\infty (a_k, b_k) \right\}
    \]
\end{defn}

\begin{prop}[Properties of the Lebesgue outer measure]
    The Lebesgue measure obeys the following properties.
    \begin{itemize}
        \item $m^*(A)$ is defined, and $m^*(A) \in [0,\infty]$ for any subset of $\R$.
        \item $m^*(\emptyset) = 0$
        \item If $A \subseteq B$, $m^*(A) \leq m^*(B)$
        \item For every sequence $A_0, A_1,\dots$, we have \[
            m^*(\bigcup_{k=0}^\infty A_k) \leq \sum_{k=0}^\infty m^*(A_k)
        \]
    \end{itemize}
\end{prop}

\begin{defn}[Outer Measure]
    A function $\mu^* : \mathcal{P} \rightarrow [0,\infty]$ is such that 
    \begin{itemize}
        \item $\mu^*(\emptyset) = 0$
        \item If $A \subseteq B$, $\mu^*(A) \leq \mu^*(B)$
        \item For every sequence $A_0, A_1,\dots$, we have \[
            \mu^*(\bigcup_{k=0}^\infty A_k) \leq \sum_{k=0}^\infty \mu^*(A_k)
        \]
    \end{itemize} 
    Then $\mu^*$ is called an \emph{outer measure} on $X$.
\end{defn}

\begin{thm}[Construction from outer measures]
    Let $\mu^*$ be an outer measure on a set $X$.  Then \[
        \siga = \{ A \subseteq X \given \mu*(S) = \mu^*(S \cap A) + \mu^*(S \cap A^c) \text{ for all } S \subseteq X \}
    \]
    Then $\siga$ is a $\sigma$-algebra.  Let $\mu(A) = \mu^*(A)$ when $A \in \siga$.  Then $\mu : \siga \rightarrow [0,\infty]$ is a measure.
\end{thm}

\begin{prop}
    Let $\mu^*$ be an outer measure, and let $\siga$ be the \sig defined in the last theorem.  Let $A \subseteq X$ satisfy $\mu^*(A) = 0$.  Then $A \in \siga$, and so $\mu(A)$ is defined, and equals 0.   
\end{prop}

\begin{defn}[Null set]
    If $\mu: \siga \rightarrow [0,\infty]$ is any measure, then a set $A \in \siga$ satisfying $\mu(A) = 0$ is called a null set.
\end{defn}
% subsection constructing_sigma_algebras_and_measures (end)


\subsection{Properties of the Lebesgue measure on $\R$} % (fold)
\label{sub:properties_of_the_lebesgue_measure_on_r_}
Let \[
\mathcal{M} = \{ A \subseteq \R \given m*(S) = m^*(S \cap A) + m^*(S \cap A^c) \text{ for all } S \subseteq \R \}
\]

The sets in $\mathcal{M}$ are called hte \emph{Lebesgue measurable subsets} of $\R$.  If $A \in \mathcal{M}$, the we write $m(A) = m^*(A)$.  This real number is called the \emph{Lebesgue measure of $A$}.  

We now show that this \sig $\mathcal{M}$ is very large.

\begin{thm}Let $m^*$ denote the Lebesgue outer measure on $\R$, and let $\mathcal{M}$ be the \sig of Lebesgue measurable sets.  Then 
    \begin{itemize}
        \item If $I \subseteq \R$ is an interval.  Then $m^*(I) = l(I)$.  That is, the outer measure is just its length.
        \item If $I \subseteq \R$ is an interval, then $I \in \mathcal{M}$.
    \end{itemize}
    
\end{thm}

\begin{prop}
    Any open subset of $\R$ is in $\mathcal{M}$.  Any closed subset of $\R$ is in $\mathcal{M}$.  That is, all open or closed sets in $\R$ are Lebesgue measurable.
\end{prop}

\begin{cor}
    Every Borel subset of $\R$ is contained in $\mathcal{M}$.  
\end{cor}
\begin{proof}
    $\mathcal{M}$ is a \sig which contains every open subset of $\R$.  The \sig $\mathcal{B}$ is by definition the smallest such $\sigma$-algebra.  Thus $\mathcal{B} \subseteq \mathcal{M}$.
\end{proof}

%!TEX root = /Users/ajtulloch/Documents/University 2010/MATH 3969 - Measure Theory and Fourier Analysis/Exam Notes/MATH 3969 - Master Notes.tex

\section{Measurable Functions} % (fold)
\label{sec:measurable_functions}
\newcommand{\Rbar}{\overline{\R}}

\begin{defn}[Measurable function]
    Let $\siga$ be \sig of subsets of a set $X$.  A function $f: X \rightarrow \Rbar$, is called \emph{measurable} (or $\siga$-measurable) if for every $\alpha \in \R$, the set \[
        \{ x \in X \given f(x) > \alpha \}
    \] is in $\siga$.
\end{defn}

\begin{defn}[Indicator function].  Let $S \subset X$.  We define the \emph{indicator function} of $S$ to be the function $1_S : X \rightarrow \R$ given by 

    \[ 1_S(x) = \begin{cases}
        1 \quad \text{if $x \in S$}\\
        0 \quad \text{if $x \notin S$}
    \end{cases}\]
\end{defn}

\begin{prop}
    Let $S \subset X$.  Then $1_S$ is measurable if and only if $S \in \siga$.  
\end{prop}
\begin{proof}
    Let $\alpha \in \R$.  Then \[
        \{ x \in X \given 1_S(x) > \alpha \} = \begin{cases}
            \emptyset \quad \text{if $\alpha \geq 1$}\\
            S \quad \text{if $0 \leq \alpha < 1$} \\
            X \quad \text{if $\alpha < 0$}
        \end{cases}
    \]
    
As $\emptyset, X$ are in $\siga$, then $1_S$ is measurable if and only if $S \in \siga$.
\end{proof}

\begin{prop}[Continuous functions are Lebesgue measurable]
    Let $f: [a,b] \rightarrow \R$ be continuous.  Then $f$ is Lebesgue measurable on $[a,b]$.  More generally, if $X \subset \R$ is in $\mathcal{M}$ and $f: X \rightarrow \R$ is continuous, then $f$ is Lebesgue measurable on $X$.
\end{prop}

\subsection{Basic properties of measurable functions} % (fold)
\label{sub:basic_properties_of_measurable_functions}

\begin{lem}
    Let $\siga$ be a \sig of subsets of a set $X$, and let $f : X \rightarrow \Rbar$ be a function.  Then $f$ is measurable if and only if it satisfies one of the following conditions.
    \begin{itemize}
        \item For each $\alpha \in \R$, the set $\{ x \in X \given f(x) > \alpha \}$ in in $\siga$.
        \item For each $\alpha \in \R$, the set $\{ x \in X \given f(x) < \alpha \}$ in in $\siga$.
        \item For each $\alpha \in \R$, the set $\{ x \in X \given f(x) \leq \alpha \}$ in in $\siga$.
        \item For each $\alpha \in \R$, the set $\{ x \in X \given f(x) \geq \alpha \}$ in in $\siga$.
    \end{itemize}
\end{lem}

\begin{prop}Let $\siga$ be a \sig of subsets of a set $X$, and let $f,g : X \rightarrow \Rbar$ be functions.  Then 
    \begin{itemize}
        \item $f + g$ is measurable (provided that $f(x) = \infty$ and $g(x) = - \infty$ or vice versa holds for no $x \in X$).
        \item $cf$ is measurable for any constant $c \in \R$.
        \item $fg$ is measurable.
        \item $f/g$ is measurable (provided that $g(x)$ is nonzero and not infinity for all $x \in X$).
    \end{itemize}
    
    Similarly, let $f_0,f_1,\dots : X \rightarrow \Rbar$.  Then 
    \begin{itemize}
        \item $\sup\{f_0,f_1,\dots\}$ and $\inf\{f_0,f_1,\dots\}$ are measurable functions.
    \end{itemize}
\end{prop}
\begin{cor}
    Let $f,g$ be measurable.  Then $\max\{f,g\}$ and $\min\{f,g\}$ are measurable functions.
\end{cor}

\begin{prop}
    Let $\siga$ be a \sig of subsets of a set $X$, and let $f_0, f_1,\dots : X \rightarrow \Rbar$ be measurable functions.  Let $f(x) = \lim_{k \rightarrow \infty} f_k(x)$ for each $x \in X$. Then $f$ is a measurable function.
\end{prop}  

% subsection basic_properties_of_measurable_functions (end)



\subsection{Simple functions} % (fold)
\label{sub:simple_functions}
\begin{defn}[Simple function]

    Let $\siga$ be a \sig of subsets of a set $X$.  A function $\phi : X \rightarrow \R$ is called \emph{simple} if it is measurable and only takes a finite number of values.
\end{defn}


\begin{prop}
    Let $\siga$ be a \sig of subsets of a set $X$, and let $f: X \rightarrow [0, \infty]$ be a nonnegative measurable function.  Then there is a sequence $(\phi_n)$ of simple functions such that 
    \begin{itemize}
        \item $0 \leq \phi_1(x) \leq \phi_2(x) \leq \dots \leq f(x)$ for all $x \in X$.
        \item $f(x) = \lim_{n \rightarrow \infty} \phi_n(x)$ for all $x \in X$.
    \end{itemize}
\end{prop}

\begin{proof}
    Define the function $\phi_n$ as follows.
    
    Let \[
        A_{n,k} = \{ x \in X \given \frac{k}{2^n} \leq f(x) < \frac{k+1}{2^n} \}
    \]
    and let $A_{n,2^n} = \{ x \in X \given f(x) \geq n \}$
    
    Then the function \[
        \phi_n  = \sum_{k=0}^{n2^n} \frac{k}{2^n}1_{A_{n,k}}
    \]
    obeys the required properties.
\end{proof}















% subsection simple_functions (end)























%!TEX root = /Users/ajtulloch/Documents/University 2010/MATH 3969 - Measure Theory and Fourier Analysis/Exam Notes/MATH 3969 - Master Notes.tex


% \newcommand{\dmu}{\, d \mu}

\section{Integration} % (fold)
\label{sec:integration}





\begin{defn}[Integration of simple functions]
    Let $\phi = \sum_{j=1}^m a_j 1_{A_j}$.  Then the \emph{integral $\int_X \phi \dmu$ of $\phi$ over $X$ with respect to $\mu$} is given by \[
        \int_X \phi \dmu = \sum_{j=1}^m a_j \mu (A_j)
    \]
\end{defn}

\begin{prop}
    Let $\phi$ and $\psi$ be nonnegative simple functions on $X$, and let $c \geq 0$ be constant.  Then \begin{itemize}
        \item $
            \int_X \phi  + \psi \dmu = \int_X \phi \dmu + \int_X \psi \dmu
    $
        \item If $ 0 \leq \psi \leq \phi $, then \[
            0 \leq \int_0 \psi \dmu \leq \int_X \psi \dmu
        \]
        \item $\int_X c \phi \dmu = c \int_X \phi \dmu$
    \end{itemize}
\end{prop}


\begin{defn}[Integral over a subset of $X$]
    Let $\phi$ be a nonnegative simple function, and let $S \subset X$ be measurable.  Then \emph{the integral of $\phi$ over $S$ with respect to $\mu$}, denoted $\int_S \phi \dmu$, is given by \[
        \int_S \phi \dmu = \int_X \phi \cdot 1_s \dmu
    \] 
\end{defn}

\subsection{Integration of nonnegative measurable functions} % (fold)
\label{sub:integration_of_nonnegative_measurable_functions}



\begin{defn}Let $f: X \rightarrow [0, \infty]$ be a nonnegative measurable function.  We define the \emph{integral $\int_X f \dmu$ of $f$ over $X$ with respect to $\mu$} by \[
    \int_X f \dmu = \sup \left\{ \int_X \phi \dmu \given \phi \text{ is simple, and } 0 \leq \phi \leq f \text{ on } X \right\}
    \]
\end{defn}

\begin{lem}
    Suppose that $f,g$ are two nonnegative measurable functions, and $0 \leq g \leq f$
 on $X$.  Then \[
    0 \leq \int_X g \dmu \leq \int_X f \dmu
 \]\end{lem}

The following is an extremely important theorem in measure theory. 
\begin{thm}[Monotone convergence theorem]
    Let $(f_k)$ be a sequence of nonnegative measurable functions on $X$.  Assume that \begin{itemize}
        \item $ 0 \leq f_0(x) \leq f_1(x) \leq \dots$ for each $x \in X$,
        \item $\lim_{k \rightarrow \infty} f_k(x) = f(x)$ for each $x \in X$.
    \end{itemize}
    When these hold, we write $f_k \nearrow f$ pointwise.  
    
    Then $f$ is measurable, and \[
        \int_X f \dmu = \lim_{ k \rightarrow \infty} \int_X f_k \dmu
    \] 
\end{thm}

\begin{cor}
    Let $f,g$ be measurable on $X$.  THen \[
        \int_X f + g \dmu = \int_X f \dmu + \int_X g \dmu
    \]
\end{cor}

\begin{thm}
    Suppose that $f_k$ is a nonnegative measurable function for $k = 0,1,\dots$.  Then \[
        \int_X \left(\sum_{k=0}^\infty f_k \right) \dmu = \sum_{k=0}^\infty \left( \int_X f_k \dmu \right)
    \]
\end{thm}

\begin{thm}
    Suppose that $X$ is a set, $\siga$ is a \sig of subsets of $X$ and $\mu: \siga \rightarrow [0,\infty]$ is a measure.  Let $f$ be nonnegative and measurable on $X$.  Then define $\mu_f : \siga \rightarrow [0,\infty]$ by \[
        \mu_f(A) = \int_A f \dmu = \int_X f \cdot 1_A \dmu
    \]
    Then $\mu_f$ is a measure.
\end{thm}

\begin{prop}
    Suppose that $f$ is nonnegative and measurable on $X,$ and suppose that $\int_X f \dmu < \infty$.  Then the set $\{ x \in X \given f(x) = \infty \}$ has measure 0.
\end{prop}


\begin{prop}
    Suppose that $X$ is a set, $\siga$ is a \sig of subsets of $X$ and $\mu: \siga \rightarrow [0,\infty]$ is a measure. Suppose that there is a set $N \in \siga$ with $\mu(N) = 0$ and suppose that some property $P$ holds for all $x \in X$ outside $N$.  Then we say that the property $P$ holds \emph{almost everywhere} or \emph{for almost all $x \in X$}.
\end{prop}

\begin{prop}
    Suppose that $f_k$ is a nonnegative measurable function on $X$ for $k = 0,1,\dots$.  Suppose that \[
        \sum_{k=0}^\infty \left(\int_X f_k \dmu \right) < \infty
    \]
Then \[
    \sum_{k=0}^\infty < \infty \text{ for almost all $x \in X$}
\]
\end{prop}


\begin{prop}
    Suppose that $f$ is a nonnegative measurable function on $X$.  Then \[
        \int_X f \dmu = 0 \iff f(x) = 0 \text{ almost everywhere.}
    \]
\end{prop}


\begin{thm}[Fatou's Lemma]
    Suppose that $f_k$ is a nonnegative measurable function on $X$, for $k = 0,1,\dots$, and that $\lim_{k \rightarrow \infty} f_k(x) = f(x)$ for each $x \in X$.  Then \[
        \int_X f \dmu \leq \liminf_{k \rightarrow \infty} \left( \int_X f_k \dmu \right)
    \]
    \end{thm}
% subsection integration_of_nonnegative_measurable_functions (end)



\subsection{Integration of real and complex valued functions} % (fold)
\label{sub:integration_of_real_and_complex_valued_functions}
\begin{lem}
    
    Let $f: X \rightarrow \Rbar$ be a measurable function, and let $f^+$ and $f^-$ be the positive and negative parts of $f$.  Then $f = f^+ - f^-$, and $|f| = f^+ + f^-$.  Moreover, $|f|$ is a measurable function, and
    \[
        f \text{ is integrable if and only if } \int_X |f| \dmu < \infty
    \]
\end{lem}


\begin{defn}[Integral of a complex valued function]
    Let $f = u + iv,$ where $u,v : X \rightarrow \Rbar$.  Then \[
        \int_X f \dmu = \int_X u \dmu + i \int_X v \dmu
    \]
\end{defn}
\begin{lem}
    Let $f : X \rightarrow \bar{\mathbb{C}}$.  Then $f$ is integrable if and only if $\int_X |f| \dmu < \infty$.
\end{lem}


The next theorem is probably the most important single theorem in these notes.  It has many applications, both of a theoretical and practical nature.
\begin{thm}[Dominated convergence theorem]
    Let $(f_k)$ be a sequence of real or complex valued measurable function on $X$. Assume that \begin{itemize}
        \item $\lim_{k \rightarrow \infty} f_k(x) = f(x)$
    \end{itemize}
    and that there is a measurable function $g: X \rightarrow [0,\infty]$ such that 
    \begin{itemize}
        \item $|f_k(x)| \leq g(x)$ for each $k$ and $x$, and 
    \end{itemize}
    
    \begin{itemize}
        \item $\int_X g \dmu < \infty$
    \end{itemize}
    Then \[
        \int_X f \dmu = \lim_{k \rightarrow \infty} \int_X f_k \dmu
    \]
\end{thm}



\begin{thm}[Bounded convergence theorem]
    Let $(f_k)$ be a sequence of real or complex valued measurable function on $X$.  Assume that \begin{itemize}
        \item $\lim_{k \rightarrow \infty} f_k(x) = f(x)$ for each $x \in X$,
        \item There exists a constant $M < \infty$ such that $|f_k(x)| \leq M$ for each $k$ and $x$,
        \item $\mu(X) < \infty$.
    \end{itemize}
    Then \[
        \int_X f \dmu = \lim_{k \rightarrow \infty} \int_X f_k \dmu
    \]
    \end{thm}

\begin{thm}
    Suppose that $f_k$ is a measurable real or complex valued function on $X$ for $k = 0,1,\dots$.  Suppose that we have \[
        \sum_{k = 0}^\infty \left(\int_X |f_k| \dmu \right) < \infty
    \] or, equivalently,
    \[
        \int_X \left( \sum_{k = 0}^\infty |f_k| \right) \dmu < \infty
    \]
    Then we have \[
    \int_X \left( \sum_{k = 0}^\infty f_k \right) \dmu =
    \sum_{k = 0}^\infty \left(\int_X f_k \dmu \right)   
    \]
\end{thm}



\begin{defn}[Integrable function]
    We call $f : X \rightarrow \mathbb{K}$ $\mu$ -integrable if $f$ is $\mu$ measurable and 
    
    \[
        \int_X |f| \, d \mu < \infty
    \]
    
    We set \[
        \mathcal{L}^1(X,\mathbb{K}) = \{ f : X \rightarrow \mathbb{K} \, | \, f \mu \text{-integrable} \}
    \]
\end{defn}

\begin{thm}
    $   \mathcal{L}^1(X,\mathbb{K})$ is a vector space over $\mathbb{K}$
\end{thm}


\begin{defn}[The Lebesgue-Stieltjes integral]
Let $F : \R \rightarrow \R$ be an increasing right continuous function, that is $\lim_{s \rightarrow t^+} F(s) = F(t)$ for all $t \in \R$.  Then for $A \subseteq \R$ let \[
    \mu{^\star}_F (A) = \inf \{ \sum_{k=0}^\infty (F(b_k) - F(a_k)) \, | \, A \subseteq \bigcup_{k \in \mathbb{N}} (a_k, b_k)
\]
The $\mu^\star_F$ is an outer measure inducing an inner measure on $\R$.  Then we have 
    \begin{itemize}
        \item $\mu_F$ is a Borel measure.
        \item $\mu_F((a,b]) = F(b) - F(a)$.
    \end{itemize}
    
    We then define $\int_A f \, dF = \int_A f \, d \mu_F $ as the Lebesgue-Stieltjes integral.
\end{defn}

\begin{lem}
    If $\mu$ is a finite measure on $\R$, then we define $F(t) = \mu((-\infty, t])$ as the \textbf{distribution} function of $\R$.
\end{lem}

\begin{thm}
    There is a bijection from finite measures and some class of right-continuous increasing functions.
\end{thm}

\begin{defn}[Measures from other measures]
    Let $g: X \rightarrow [0,\infty]$ be a $\mu$-measurable function.  For $A \in \mathcal{A}$ define \[
        \nu(A) = \int_A g \dmu
    \]  
Then using the monotone convergence theorem one can show that $\nu$ is a measure defined on $\siga$.  Moreover, if $f: X \rightarrow \mathbb{K}$ is $\mu$-measurable, then \[
    \int_X f \, d \nu = \int_X fg \dmu
\]
We call $g$ the \textbf{density of $\nu$ with respect to $\mu$}.
\end{defn}

\begin{prop}
    Let $f \in  \mathcal{L}^1(X,\mathbb{K})$ with respect to the Lebesgue measure.  Then \[
        \int_a^b f(x) \, dx = \lim_{t \rightarrow b^-} \int_a^t f(x) \, dx
    \]
\end{prop}

\subsection{Parameter integrals} % (fold)
\label{sub:parameter_integrals}
\newcommand{\ellone}[1]{\mathcal{L}^1(X,#1)}
\begin{defn}[Parameter integral]
    Let $(X, \siga, \mu)$ be a measure space and $Y$ a metric space.  Suppose that $f : X \times Y \rightarrow \mathbb{K}$ is such that 
     \begin{itemize}
        \item $x \mapsto f(x,y)$ is $\mu$-integrable for all $y \in Y$,
        \item $y \mapsto f(x,y)$ is continuous at $y_0$ for almost all $x \in X$,
        \item there exists $g \in \ellone{\R}$ such that \[
            |f(x,y)| \leq g(x)
        \] for almost all $x \in X$.  

     \end{itemize}
        Define $F(y) = \int_X f(x,y) \dmu(x)$.  Then $F$ is continuous at $y_0 \in Y$.
    
\end{defn}


\begin{thm}[Differentiation of parameter integrals.]
    Let $(X, \siga, \mu)$ be a measure space and $L \subset \R$ an interval.  Suppose that $f: X \times L \rightarrow \R$ is such that 
    \begin{itemize}
        \item $x \mapsto f(x,y)$ is $\mu$-integrable for all $y \in Y$,
        \item $\frac{\partial}{\partial t} f(x,t)$ exists for all $t \in L$, for almost all $x \in X$, and is continuous ,
        \item there exits $g \in \ellone{\R}$ with $|\frac{\partial}{\partial t} f(x,t)| < g(x)$ for almost all $x \in X$ and all $t \in L$.   
        
    \end{itemize}
    
    Define $F(t) = \int_X f(x,t) \dmu(x)$.  Then $f : L \rightarrow \mathbb{K}$ is differentiable and \[
        F'(t) = \int_X \frac{\partial}{\partial t} f(x,t) \dmu(x)
    \]
\end{thm}

% subsection integration_of_real_and_complex_valued_functions (end)






% section integration (end)




















%!TEX root = /Users/ajtulloch/Documents/University 2010/MATH 3969 - Measure Theory and Fourier Analysis/Exam Notes/MATH 3969 - Master Notes.tex

\newcommand{\K}{\mathbb{K}}
\newcommand{\ellp}[1]{\mathcal{L}^{#1}(X)}
\newcommand{\Lp}[1]{L^{#1}(X)}

\section{The $L^p$-spaces} % (fold)
\label{sec:the_l_p_spaces}



\begin{defn}[$L^p$-spaces]
    Let $1 \leq p < \infty$ and $f:X \rightarrow \K$ measurable. We call \[
        \| f \|_p = \left(\int_X |f|^p \dmu\right)^\frac{1}{p}
    \]
    the \emph{$L^p$-norm} of $f$.  We set \[
        \ellp{p} = \{ f: X \rightarrow \K \, | \, f \text{ measurable}, \| f \|_p < \infty
    \]  
\end{defn}


\begin{thm}[H\"older's inequality]
    Let $p,q \in [1,\infty]$ such that $\frac{1}{p}+\frac{1}{q} = 1$.  If $f \in \ellp{p}$ and $g \in \ellp{q}$, then \[
        | \int_X fg \dmu | \leq \|f\|_p \|g\|_q
    \]
\end{thm}

\begin{prop}[Minkowski's inequality]
    If $f,g \in \ell{p}$, $1\leq p \leq \infty$, then \[
        \|f + g \|_p \leq \|f \|_p + \|g \|_p
    \]
\end{prop}

\begin{defn}[$L^p$-spaces]
    Let $f \sim g$ if $f = g$ almost everywhere.  Denote the equivalence class of $f$ by $[f]$.  Then \[
        \Lp{p} = \{ [f] \, | \, f \in \ellp{p} \} 
    \]
\end{defn}

\begin{defn}[Cauchy sequence]
    A sequence $(f_n) \in \Lp{p}$ is called \textbf{Cauchy} if for every $\epsilon > 0$ there exists $n_0 \in \mathbb{N}$ such that \[
        \| f_n - f_m \|_p < \epsilon
    \] for all $n,m > n_0$.
\end{defn}

\begin{thm}[Completeness of $\Lp{p}$]
    Let $(f_n)$ be a sequence in $\Lp{p}$.  Then $(f_n)$ converges in $\Lp{p}$ if and only if $(f_n)$ is a Cauchy sequence.
\end{thm}

\begin{rem}
    Introducing the metric $d(f,g) = \|f-g\|_p$, we have that $\Lp(X)$ is a \textbf{complete normed space} or a \textbf{Banach space}. If $p = 2$, then $\|f \|_2 $ is induced by an inner product - hence $\Lp{2}$ is a \textbf{complete inner product space}, or a \textbf{Hilbert space.}
\end{rem}


\begin{prop}
    Suppose that $f_n, f \in \ellp{p}$ with $\|f_n - f \| \rightarrow 0$. Then there exists a subsequence $(f_{n_k})$ with $f_{n_k}$ converging pointwise to $f$ for almost every $x \in X$.
\end{prop}


\begin{thm}
    The simple functions are dense in $\Lp{p}$ for $1 \leq p < \infty$.
    
    In $\R^N$ and the Lebesgue measure, we can modify the statement to the simple function with bounded support are dense in $\R^N$. 
\end{thm}


\begin{thm}
    For $1 \leq p < \infty$
    \[
        \text{span} \{1_U \, | \, U \subseteq \R^N \text{ open and bounded} \}
    \] is dense in $L^p(\R^N)$.  
    
    We can also use bounded rectangles in the place of open bounded sets here.
\end{thm}

\begin{defn}[Essential supremum]
    Let $(X, \siga, \mu)$ be a measure space and $f : X \rightarrow \R$ $\mu$-measurable.  We call $\text{ess-sup} f(x) = \inf \{ t \in \R \, | \, \mu(\{ x \in X \, | \, f(x) > t \} ) = 0 \}$ the \emph{essential supremum} of $f$.  The essential supremum of $|f|$ is denoted $\| f \|_\infty$
\end{defn}

\begin{thm}[Completeness of $\Lp{\infty}$]
    $\Lp(\infty)$ is a complete normed space.
\end{thm}

\begin{lem}
    H\"older's inequality holds for $p = 1, q = \infty$.  That is, \[
        | \inf_X fg \dmu | \leq \|f \|_p \|g \|_q
    \]
\end{lem}

\begin{lem}
    If $\mu(X) < \infty$, then $\lim_{p \rightarrow \infty} \|u\|_p = \|f\|_\infty$
\end{lem}


\subsection{Fubini's Theorem} % (fold)
\label{sub:fubini_s_theorem}

\begin{thm}[Tonelli]\label{tonelli}
    Suppose that $f: \R^n \times \R^m \rightarrow [0,\infty]$ is measurable. Then there exist sets $N \subseteq \R^n$ and $M \subseteq \R^m$ of measure zero such that 
    
    \begin{enumerate}[(i)]
        \item $x \mapsto f(x,y)$ is measurable for all $y \in \R^m - M$,
        \item $y \mapsto \int_{\R^n} f(x,y) \, dx$ is measurable,
        \item $y \mapsto f(x,y)$ is measurable for all $x \in \R^m - N$,
        \item $x \mapsto \int_{\R^m} f(x,y) \, dy$ is measurable,
        \item  \begin{align*}
            &\int_{\R^n \times \R^m}f(x,y)\, d(x,y) = \\
            &\int_{R^m-M} \left( \int_\R^n f(x,y) dx \right) dy = \\
            &\int_{R^n-N} \left( \int_\R^m f(x,y) dy \right) dx
        \end{align*}
    \end{enumerate}
\end{thm}

\begin{thm}[Fubini]
    Suppose that $f: \R^n \times \R^m \rightarrow [0,\infty]$ is measurable. Let $N,M$ be the sets from Theorem \ref{tonelli} applied to the function $|f| $ such that (v) holds with $f$ replaced with $|f|$.  Assume that one of these integrals is finite - and hence all of them.  Then there exists sets $N_1$ of $\R^n$ and $M_1$ of $\R^m$ such that $(i) - (v)$ of Theorem \ref{tonelli} hold with $N,M$ replaced with $N_1, M_1$.
\end{thm}


\begin{defn}[Complete measure space]
    Let $(X,\siga, \mu)$ be a measure space.  We call the measure $\mu$ \emph{complete} if whenever $A \in \siga$ has measure $0$, then any subset of $A$ is in $\siga$, (and has measure 0).
\end{defn}
\begin{defn}[$\sigma$-finite measure space]
    
\end{defn}



% subsection fubini_s_theorem (end)

% section the_l_p_spaces (end)




%!TEX root = /Users/ajtulloch/Documents/University 2010/MATH 3969 - Measure Theory and Fourier Analysis/Exam Notes/MATH 3969 - Master Notes.tex

\section{Convolution} % (fold)
\label{sec:convolution}


\begin{defn}[Translation of a function]
    Let $f: \R^N \rightarrow \Com$ be a function and $t \in \R^N$ a fixed vector. We define the translation operator $\tau_t$ by \[
        \tau_t f(x) = f(x-t)
    \]
\end{defn}

\begin{thm}[Continuity of translation]
    Let $1 \leq p < \infty$ and $f \in L^p(\R^N)$.  Then \[
        \lim_{t \rightarrow 0} \| \tau_t f - f \|_p = 0
    \]
\end{thm}
\begin{rem}
    This does not hold if $p = \infty$.
\end{rem}

\begin{lem}
    Let $f : \R^N \rightarrow \Com$ be measurable and set 
    \begin{align*}
        F_1(x,y) &= f(x) \\
        F_2(x,y) &= f(y-x)
    \end{align*}
    
    Then $F_1, F_2 : \R^N \times \R^N \rightarrow \Com$ are measurable.
\end{lem}

\begin{defn}[Convolution]
    Let $f,g : \R^N \rightarrow \Com$ be measurable.  We define the \textbf{convolution} $f \star g : \R^N \rightarrow \Com$ by \[
        (f \star g) (x) = \int_{\R^N} f(x-y) g(y) \, dy
    \]
    wherever the integral exists
\end{defn}

\begin{defn}[Convex function]
    A function $\phi : (a,b) \rightarrow \R$ is called \textbf{convex} if \[
        \phi(\lambda s + (1-\lambda)t) \leq \lambda \phi(s) + (1-\lambda) \phi(t)
    \] for all $s,t \in (a,b)$ and all $\lambda \in (0,1)$
\end{defn}

\begin{lem}
    This is equivalent to the condition\[
        \frac{\phi(t) - \phi(s)}{t-s} \leq \frac{\phi(u) - \phi(t)}{u - t}
    \] whenever $a < s < t < u < b$.
\end{lem}

\begin{thm}[Jensen's inequality in $\ellp{p}$-spaces]
    Let $f \in \ellp{p}$, $1\leq p < \infty$, and let $g \in \ellp{1}$.  Then \[
        \left(\int_X |fg| \dmu \right)^p \leq \|g\|^{p-1}_1 \int_X |f|^p |g| \dmu
    \]
\end{thm}

\begin{thm}[Young's inequality]
    Let $1 \leq p \leq \infty$.  If $f \in L^p(\R^N, \Com)$ and $g \in L^1(\R^N, \Com)$, then $f \star g$ exists almost everywhere and $f \star g \in L^p(\R^N, \Com)$.  Moverover, \[
        \|f \star g \|_p \leq \|f\|_p \|g \|_1
    \]
\end{thm}

\begin{thm}
    Let $1 \leq p, q \leq \infty$ with $\frac{1}{p} + \frac{1}{q} = 1$.  If $f \in L^p(\R^N)$ and $g \in L^q(\R^N)$, then \[
        f \star g \in BC(\R^N)
    \] where $BC$ is the vector space of bounded continuous functions.
\end{thm}


\subsection{Approximate identities} % (fold)
\label{sub:approximate_identities}

\begin{defn}[Approximate identity]
    Let $\phi: \R^N \rightarrow [0,\infty)$ be measurable with \[
        \int_{\R^N} \phi \, dx = 1
    \] and set $\phi_n(x) = n^N \phi(n x)$ for all $x \in \R^N$ and $n \in \mathbb{N}$.  Then $(\phi_n)$ is called an \textbf{approximate identity}
\end{defn}

\begin{thm}
    Let $(\phi_n)$ be an approximate identity and $f \in L^p(\R^N), 1 \leq p < \infty$.  Then \[
        f \star \phi_n \rightarrow f
    \] in $L^p(\R^N)$ as $n \rightarrow \infty$
\end{thm}

\begin{thm}
    Let $f \in L^\infty(\R^N)$ and $(\phi_n)$ an approximate identity.  If $f$ is continuous at $x$, then \[
        f(x) = \lim_{n \rightarrow \infty} (f \star \phi_n) (x)
    \]
\end{thm}

% subsection approximate_identities (end)


\begin{defn}[Test function]
    Let $U \subseteq \R^N$ be open.  We let \[
        C^\infty (U, \K) = \{ f: U \rightarrow \K \, | \, f \text{ has partial derivatives of all orders} \}
    \]
    and \[
        C_c^\infty(U, \K) = \{ f \in C^\infty(U, \K) \, | \, \text{supp}(f) \subseteq U, \text{supp($f$) compact}
    \]
    
    The functions in $C_c^\infty(U, \K)$ are called \textbf{test functions} on $U$.
\end{defn}

\begin{prop}
    Let $f : \R^N \rightarrow \K$ be measurable such that $f \in \mathcal{L}^1(B)$ for every bounded set $B \subseteq \R^N$.  If $\phi \in C_c^\infty(\R^N)$, then $f \star \phi \in C^\infty(\R^N)$ and \[
        \frac{\partial}{\partial x_i} (f \star \phi) = f \star \frac{\partial \phi}{\partial x_i}
    \]
\end{prop}

\begin{thm}
    Let $U \subseteq \R^N$ open and $1 \leq p < \infty$.  Then $C_c^\infty(U)$ is dense in $L^p(U)$.
\end{thm}
\begin{rem}
    The above proposition does not hold for $p = \infty$.
\end{rem}
% section convolution (end)



















%!TEX root = /Users/ajtulloch/Documents/University 2010/MATH 3969 - Measure Theory and Fourier Analysis/Exam Notes/MATH 3969 - Master Notes.tex

\renewcommand{\hat}{\widehat}

\section{The Fourier Transform} % (fold)
\label{sec:the_fourier_transform}

\begin{defn}[Fourier transform]
    Let $f \in L^1(\R^N, \Com)$.  We call \[
        \hat{f}(t) = \int_{\R^N} f(x) e^{-2 \pi i x \cdot t} \, dx
    \]
\end{defn}

\begin{thm}
    We have 
    \begin{itemize}
        \item $\hat{f} : \R^N \rightarrow \Com$ is continuous,
        \item $\|\hat f \|_\infty \leq \|f \|_1$
    \end{itemize}
\end{thm}

\begin{prop}
    Let $\phi(x) = e^{- \pi |x|^2}$.  Then $\|\phi \|_1 = 1$ and $\hat \phi = \phi$.
\end{prop}

\begin{prop}
    Let $f \in L^1(\R^N, \Com), x_0 \in \R^N$ and $\alpha \in R, \alpha > 0$. 
    \begin{enumerate}[(i)]
        \item If $g(x) = f(x - x_0)$, then $\hat g (t) = e^{-2 \pi i x_0 \cdot t} \hat f (t)$,
        \item If $g(x) = f(\alpha x)$, then $\hat g (t) = \frac{1}{\alpha^N} \hat f \left( \frac{x}{\alpha}\right)$,
        \item \vspace{0.1cm} If $g(x) = \overline{f(-x)}$, then $\hat g (t) = \overline{\hat f(t)}$
    \end{enumerate}
\end{prop}

\begin{defn}
    Let $C_0(\R^N, \K) = \{ f \in C(\R^N, K) \, | \, \lim_{|x| \rightarrow \infty} f(x) = 0 \}$, the set of continuous functions vanishing at infinity
\end{defn}

\begin{thm}[Riemann-Lebesgue]
    If $f \in L^1(\R^N, \Com)$, then $\hat f \in C_0(\R^N, \Com)$ and $\| \hat f \|_\infty \leq \|f \|_1$.
\end{thm}

\begin{thm}
    If $f,g \in L^1(\R^N, \Com)$ then $f \star g \in L^1(\R^N, \Com)$ and \[
        \hat{f \star g} = \hat f \, \hat g
    \]
\end{thm}
\begin{prop}
    Let $f,g \in L^1(\R^N, \Com)$.  Then \[
        \int_{\R^N} \hat f g \, dx = \int_{\R^N} f \hat g \, dx
    \]
\end{prop}

\begin{lem}
    Let $\phi(x) = e^{- \pi |x|^2}$ and $\phi_n(x) = n^N \phi(n x)$.  Then \[
        \int_{\R^N} \hat f (t) e^{2 \pi i x \cdot t} \phi\left(\frac{t}{n} \right) \, dt = (f \star \phi_n) (x)
    \]
    for all $f \in L^1(\R^N,\Com), x \in \R^N$, and $n \in \mathbb{N}$.
\end{lem}

\begin{thm}[Fourier inversion formula]
    Let $f \in L^1(\R^N, \Com)$. Then 
    \begin{enumerate}[(i)]
        \item \[
            \lim_{n \rightarrow \infty} \int_{\R^N} \hat f (t) e^{2 \pi i x \cdot t} e^{- \pi \frac{|t|^2}{n^2}} \, dt = f
        \] in $L^1(\R^N, \Com)$.
        \item If $f$ is continuous at $x$, then \[
            \lim_{n \rightarrow \infty} \int_{\R^N} \hat f (t) e^{2 \pi i x \cdot t} e^{- \pi \frac{|t|^2}{n^2}} \, dt = f(x)
        \]
    \end{enumerate}
\end{thm}

\begin{cor}
    Let $f,g \in L^1(\R^N)$ with $\hat f = \hat g$. Then $f = g$ almost everywhere. 
\end{cor}

\subsection{The Fourier transform on $L^2(\R^N)$} % (fold)
\label{sub:the_fourier_transform_on_l_2_r_n_}
    We have defined the Fourier transform $\hat f$ with $f \in L^1(\R^N)$.  We  have that $C_c^\infty(\R^N)$ is dense in $L^2(\R^N)$ as well as in $L^1(\R^N)$, so in particular $L^2(\R^N) \cap L^1(\R^N)$ is dense in $L^2(\R^N)$. We can use this to extend the Fourier transform to $L^2(\R^N)$.  The key for doing so is the following theorem.  
    \begin{thm}[Plancherel]
        Let $f \in L^2(\R^N) \cap L^1(\R^N)$. Then $\| \hat f \|_2 = \| f \|_2$. 
    \end{thm}

\begin{prop}
    There is a unique coninuous linear operator \[
        \mathcal{F} : L^2(\R^N) \rightarrow L^2(\R^N)
    \] such that $\mathcal{F} f = \hat f$ for all $f \in L^2(\R^N) \cap L^1(\R^N)$.  Moreover, $\| f \|_2 = \| \mathcal{F} f \|_2$ for all $f \in L^2(\R^N)$.
\end{prop}

\begin{rem}
    We use the notation $\hat f = \mathcal{F} f$ for $f \in L^2(\R^N)$.  
\end{rem}
\begin{rem}
    Let $\phi_n : \R^N \rightarrow [0,1]$ such that $\phi_n \in L^2(\R^N)$ and $\phi_n(x) \rightarrow 1$ for all $x \in \R^N$. If $f \in L^2(\R^N)$, then \[
        \hat f = \lim_{n \rightarrow \infty} \int_{\R^N} f(x) \phi_n(x) e^{-2 \pi i x \cdot t} \, dx
    \]
    Common choices for $\phi_n$ are 
    \begin{itemize}
        \item $\phi_n(x) = 1_{B(0,n)},$
        \item $\phi_n(x) = e^{- \pi \frac{|x|^2}{n^2}}$
    \end{itemize}
\end{rem}

\begin{thm}
    Let $f \in L^2(\R^N, \Com)$.  Then \[
        \lim_{n \rightarrow \infty} \int_{\R^N} \hat f (t) e^{2 \pi i x \cdot t} e^{- \pi \frac{|t|^2}{n^2}} \, dt = f
    \] in $L^2(\R^N, \Com)$.
\end{thm}

\begin{thm}
    $\mathcal{F} : L^2(\R^N) \rightarrow L^2(\R^N)$ is bijective with $(\mathcal{F}^{-1} f) (x) = (\mathcal{F} f) (-x)$ for all $f \in L^2(\R^N)
$.
\end{thm}


\begin{rem}
    Let \[
        \langle f, g \rangle = \int_{\R^N} f(x) \overline{g(x)} \, dx
    \] denote the inner product on $L^2(\R^N)$.  Then the above theorem implies \[
        \langle \hat f, \hat g \rangle = \langle f, g \rangle.
    \]
    Moverover, by approximating $f,g$ by functions in $L^2(\R^N) \cap L^1(\R^N)$ we also have \[
        \langle \hat f, g \rangle = \langle f, \overline{\hat g} \rangle
    \]
\end{rem}

% subsection the_fourier_transform_on_l_2_r_n_ (end)
% section the_fourier_transform (end)

















%!TEX root = /Users/ajtulloch/Documents/University 2010/MATH 3969 - Measure Theory and Fourier Analysis/Exam Notes/MATH 3969 - Master Notes.tex

\section{The Radon-Nikodym Theorem} % (fold)
\label{sec:the_radon_nikodym_theorem}

\subsection{The Reisz representation theorem} % (fold)
\label{sub:the_reisz_representation_theorem}

% subsection the_reisz_representation_theorem (end)

Let $H$ be an inner product space with inner product $(x|v)$.  Then $H$ is a normed space with norm \[
    |u \| = \sqrt{(u | u)}
\]
We call $H$ a \textbf{Hilbert space} if $H$ is complete with respect to $\| . \|$, that is, every Cauchy sequence in $H$ converges. 


\begin{thm}[Projections]
    Let $H$ be a Hilbert space and $M \subseteq H$ a closed subspace of $H$.  Let $u \in H$.  Then there exists $m_0 \in M$ such that \[
        \| u - m_0 \| = \min_{m \in M} \| u - m \|
    \]
    Moreover, \[
        (u - m_o | m) = 0
    \] for all $m \in M$.
\end{thm}

\begin{rem}
    Fix $g \in H$ and consider the function $\phi_g : H \rightarrow \K$ given by \[
        \phi_g(f) = (f | g)
    \] 
    Then $\phi_g$ is linear, and by the Cauchy-Swartz inequality, \[
        | \phi_g(f) = |(f | g) | \leq \|f \| \|g \|
    \] for all $f \in H$.  We say $\phi_g$ is a bounded linear functional on $H$. 
\end{rem}

\begin{defn}
    Let $H$ be a Hilbert space.  We call a linear operator $\phi: H \rightarrow \K$ a \textbf{bounded linear functional} on $H$ if there exists $M > 0$ such that \[
        | \phi(f) | \leq M \| f \|
    \] for all $f \in H$.
\end{defn}

\begin{thm}[Riesz representation theorem]
    Let $H$ be a Hilbert space over $\K$ and $\phi(H \rightarrow K)$ a bounded linear function.  Then there exists $g \in H$ such that \[
        \phi(f) = (f | g)
    \] for all $f \in H$.
\end{thm}

\subsection{The Radon-Nikodym Theorem} % (fold)
\label{sub:the_radon_nikodym_theorem}
 Suppose that $\mu$ is a measure defined on the $\sigma$-algebra $\siga$ of subsets of $X.$  Given a measurable function $g : X \rightarrow [0,\infty]$ we define \[
    \nu(A) = \int_A g \dmu
 \]
Then $\nu$ is a measure defined on the $\sigma$-algebra $\siga$.

The converse does not necessarily hold - that is, given two measures $\mu$ and $\nu$ on a $\sigma$-algebra $\siga$, there is not always a measurable function $g : X \rightarrow [0,\infty]$ such that the above equation holds.

\begin{defn}[Absolute continuity]
    Let $\nu, \mu$ be the measures defined on a $\sigma$-algebra $\siga$. We call \textbf{$\nu$ absolutely continuous with respect to $\mu$} if $\nu(A) = 0$ whenever $\mu(A) = 0$.  In that case, we write $\nu << \mu$.
\end{defn}

\begin{prop}
    Let $\mu, \nu$ be measures defined on a $\sigma$-albegra.  Suppose that $\mu(X), \nu(X) < \infty$.  Set $\lambda = \mu + \nu$.  Then there exists a measurable function $h : X \rightarrow [0,\infty] $ such that \[
        \int_X f \, d \nu = \int_X f h \, d \lambda
    \] for all $f \in L^2(X, \lambda)$
\end{prop}

\begin{thm}[Radon-Nikodym]
    Let $\mu,\nu$ be measures defined on a $\sigma$-algebra.  Suppose that $\nu$ and $\mu$ are $\sigma$-finite and that $v << u$.  Then there exits a measurable function $g : X \rightarrow [0,\infty)$ such that $\nu(A) = \int_A g \dmu$ for all $A \in \siga$. 
\end{thm}

Formally we can write \[
    \int_X f \, d \nu = \int_X f \frac{d \nu}{d\mu} \dmu
\] if we define $g = \frac{d \nu}{d\mu}$, where $g$ is the density function from the Radon-Nikodym theorem.

\begin{rem}
    If $g$ is the function in the Radon-Nikodym theorem, it is not hard to show that \[
        \int_X f \, d \nu = \int_X f g \dmu
    \] for all $f \in L^1(X, \nu)$.
\end{rem}
% subsection the_radon_nikodym_theorem (end)

% section the_radon_nikodym_theorem (end)












%!TEX root = /Users/ajtulloch/Documents/University 2010/MATH 3969 - Measure Theory and Fourier Analysis/Exam Notes/MATH 3969 - Master Notes.tex


\section{Probability Theory} % (fold)
\label{sec:probability_theory}

\begin{defn}[Random variable]
    Let $(\Omega, \siga, P)$ be a probability space.  A $\siga$-measurable function \[
        X : \Sigma \rightarrow \R
    \] is called a \textbf{random variable}. 
\end{defn}

\begin{defn}
    Let $X : \Sigma \rightarrow \R$ a random variable.  We say that $X$ has \emph{finite expectation} if $X \in L^1(\Sigma)$ and call \[
        E[X] = \int_\Sigma X \, dP
    \] the \textbf{expectation} of $X$.  
\end{defn}

\begin{defn}[Distrbution]
    For every Borel set $A \subseteq \R$ we define \[
        P_X[A] = P[ \{ \omega \in \Omega | X(\omega) \in A\}] = P[X \in A]
    \]
\end{defn}

Since $X$ is measurable, $X^{-1}[A]$ is measurable for all Borel sets $A \subseteq \R$. 

\begin{defn}[Distribution]
    Let $X$ be a random variable. The Borel measure defined above is called the \textbf{distribution} of $X$.  The function \[
        F(t) = P_X\left[(-\infty,t] \right] = P[X \leq t]
    \] is called the \textbf{distribution function} of $X$.
\end{defn}

\begin{lem}
    Let $X$ be a random variable and let $f: \R \rightarrow \R$ be Borel measurable.  Then \[
        \int_\Sigma f \circ X \, dP = \int_\R f \, dP_X
    \]
\end{lem}

\subsection{Conditional expectation} % (fold)
\label{sub:conditional_expectation}

\begin{defn}[Conditional expectation]
    Let $X: \Omega \rightarrow \R$ a random variable with finite expectation.  Let $\mathcal{A}_0$ be a $\sigma$-algebra with $\siga_0 \subseteq \siga$. We call \[
        X_0 : \Sigma \rightarrow \R
    \] a \textbf{conditional expectation} given $\siga_0$ if 
    \begin{itemize}
        \item $X_0$ is $\siga_0$-measurable
        \item $\int_A X_0 \, dP = \int_A X \, dP$ for all $A \subseteq \siga_0$.
    \end{itemize}
    We write $X_0 = E[X | \siga_0]$
\end{defn}

\begin{thm}
    Let $X$ be a random variable with finite expectation.  If $\siga_0$ is a $\sigma$-algebra with $\siga_0 \subseteq \siga$, then the conditional expectation $X_0 = E[X | \siga_0]$ exists and is essentially unique.
\end{thm}

\begin{rem}
    \begin{itemize}
        \item If $X$ is $\siga_0$-measurable, then $X = E[X | \siga_0]$ almost everywhere.
        \item If we set $\siga_0 = \{ \phi, \Omega \}$, then \[
            E[X | \siga_0] = E[X]
        \] 
    \end{itemize}
\end{rem}



% subsection conditional_expectation (end)
% section probability_theory (end)

\end{document}